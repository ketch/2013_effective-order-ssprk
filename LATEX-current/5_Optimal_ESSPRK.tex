\section{Optimal effective order explicit SSP Runge--Kutta schemes}\label{sec:optimal_ESSPRK}
In this section, we use the SSP theory and Butcher's theory of effective
order (Sections \ref{sec:SSP} and \ref{sec:Algebraic_RK}) to find
optimal explicit SSP Runge--Kutta schemes with prescribed effective
order and classical order.
According to Theorem~\ref{the:no_SSP_5}, there are no explicit SSPRK methods of
effective order five, and therefore we need only consider methods with
effective order up to four.

Recall from Section~\ref{sec:Algebraic_RK} that the methods of
effective order involve a main method $M$ as well as starting and
stopping methods $S$ and $S^{-1}$.
In Section~\ref{subsec:starting_stopping} we introduce a novel approach
to construction of starting and stopping methods in order to allow
them to be SSP.

We denote by ESSPRK($s,q,p$) an $s$-stage explicit SSP Runge--Kutta
method of effective order $q$ and classical order $p$.
Also we write SSPRK($s,q$) for an $s$-stage explicit SSP Runge--Kutta
method of order $q$.

\subsection{The main method}\label{subsec:main_method}

Our search for methods of effective order is carried out in two
steps, first searching for optimal main methods $M$ and then for
possible corresponding methods $S$ and $S^{-1}$.
For a given number of stages, effective order, and classical order,
our aim is thus to find an optimal main method, meaning one with the 
largest possible SSP coefficient $\sspcoef$.

To find a method ESSPRK($s,q,p$) with Butcher tableau $(A, \bm{b},
\bm{c})$, we consider the optimization problem~\eqref{eq:SSP_opt} 
with $\Phi(K)$ representing the conditions for effective order
$q$ and classical order $p$ (as per Table~\ref{tab:effective_OCs}).
The methods are found through numerical search, using 
\textsc{Matlab}'s optimization toolbox.
Specifically, we use \texttt{fmincon} with a sequential quadratic 
programming approach \cite{Ketcheson2008, Ketcheson/Macdonald/Gottlieb:2009}.
This process does not guarantee a global minimizer, so many searches 
from random initial guesses are performed to help ensure the method with
the largest possible SSP coefficient is found.

\subsubsection{Optimal SSP coefficients}\label{subsubsec:optimal_SSP_coeff}
Useful bounds on the optimal SSP coefficient can be obtained 
by considering an important relaxation. 
In the relaxed problem, the method is required to be accurate and strong 
stability preserving only for linear, constant-coefficient initial value problems. 
This leads to a reduced set of order conditions and a relaxed absolute 
monotonicity condition \cite{Kraaijevanger1986,Ketcheson2008,ketcheson2009a}.
We denote the maximal SSP coefficient for linear problems
(maximized over all methods with order $q$ and $s$ stages) by $\clin$.

Let $\sspcoef_{s,q}$ denote the maximal SSP coefficient (relevant to
non-linear problems) over all methods of $s$ stages with order $q$.  Let
$\sspcoef_{s,q,p}$ denote the object of our study, i.e. the maximal SSP
coefficient (relevant to non-linear problems) over all methods of $s$ stages
with effective order $q$ and classical order $p$.
From Remark~\ref{rem:talltrees} and the fact that the ESSPRK($s,q,p$) methods
form a super class of the SSPRK($s,q$) methods, we have
\begin{align} \label{ineq:clin}
        \sspcoef_{s,q} \le \sspcoef_{s,q,p} \le\clin.
\end{align}

The effective SSP coefficients for methods with up to eleven stages are shown in 
Table~\ref{tab:eff_SSP_coeff}.
Recall from Section~\ref{sec:ExRK_barrier} that $q=5$ implies a zero
SSP coefficient and from Section~\ref{sec:Algebraic_RK} that for
$q=1,2$, the class of explicit Runge--Kutta methods of effective order
is the simply the class of explicit Runge--Kutta methods.  % (of orders one and two)
Therefore we consider only methods of effective order $q=3$ and $q=4$.
Exact optimal values of $\clin$ are known for many classes of methods; for
example see \cite{Kraaijevanger1986,Ketcheson2008,ketcheson2009a}.
Those results and \eqref{ineq:clin} allow us to determine the optimal value
of $\sspcoef_{s,q,p}$ {\em a priori} for the cases $q=3$ (for any $s$) and
for $q=4,s=10$, since in those cases we have $\sspcoef_{s,q}=\clin$.

\begin{table}
    \centering
    \begin{tabular}{|c|c|ccccccccccc|}
        \hline
        \multicolumn{2}{|c|}{\backslashbox{\hspace{2pt}\vspace{1pt}$q\,,\,p$}{\vspace{-5.5pt}$s$}} & $1$ & $2$ & $3$ & $4$ & $5$ & $6$ & $7$ & $8$ & $9$ & $10$ & $11$ \\
        \hline
        $q = 3$ & $p = 2$ & $-$ & $-$ & $\bf 0.33$ & $\bf 0.50$ & $\bf 0.53$ & $\bf 0.59$ & $\bf 0.61$ & $\bf 0.64$ & $\bf 0.67$ & $\bf 0.68$ & $\bf 0.69$\\
        \hline
        $q = 4$ & $p = 2$ & $-$ & $-$ & $-$ & $0.22$ & $0.39$ & $\bf 0.44$ & $\bf 0.50$ & $\bf 0.54$ & $\bf 0.57$ & $\bf 0.60$ & $\bf 0.62$ \\
        \hline
        $q = 4$ & $p = 3$ & $-$ & $-$ & $-$ & $0.19$ & $0.37$ & $0.43$ & $\bf 0.50$ & $\bf 0.54$ & $\bf 0.57$ & $\bf 0.60$ & $\bf 0.62$ \\
        \hline
    \end{tabular}
    \caption{Effective SSP coefficients $ \ceff = \sspcoef/s$ of best known explicit effective 
    		order ESSPRK($s,q,p$) methods. 
    		Entries in bold achieve the bound $\clin$ given by the linear SSP coefficient and are therefore optimal. 
    		If no positive $\sspcoef$ can be found, we use ``$-$'' to indicate non-existence. 
    		The optimal fourth-order linear SSP coefficients are $\sspcoef^{\textnormal{lin}}_{4,4}=0.25$,
    		$\sspcoef^{\textnormal{lin}}_{5,4}=0.40$ and $\sspcoef^{\textnormal{lin}}_{6,4}=0.44$.}
    \label{tab:eff_SSP_coeff}
\end{table}

\subsubsection{Effective order three methods}\label{subsubsec:3rd_ESSPRK}
Since $\sspcoef_{s,q}=\clin$ for $q=3$, the optimal effective order three methods
have SSP coefficients equal to the corresponding optimal classical order three methods.
In the cases of three and four stages, we are able to determine exact coefficients of
families of optimal effective order methods.
\begin{theorem}\label{thm:ESSPRK(3,3,2)}
	A family of optimal three-stage, effective order three SSP Runge--Kutta 
	methods of classical order two, with SSP coefficient $\sspcoef=1$, is given by
    \begin{displaymath}
    		\begin{split}
    			\bm{Y}_1 &= \bm{u}^n, \\
    			\bm{Y}_2 &= \bm{u}^n + \Dt\bm{F}(\bm{Y}_1), \\
    			\bm{Y}_3 &= \bm{u}^n + \gamma\Dt\bm{F}(\bm{Y}_1) + \gamma\Dt\bm{F}(\bm{Y}_2), \\
    			\bm{u}^{n+1} &= \bm{u}^n + \frac{5\gamma-1}{6\gamma}\Dt\bm{F}(\bm{Y}_1) + \frac{1}{6}\Dt\bm{F}(\bm{Y}_2) + \frac{1}{6\gamma}\Dt\bm{F}(\bm{Y}_3),
        \end{split}
    \end{displaymath}
    where $1/4 \leq \gamma \leq 1$ is a free parameter.
    %We refer to the above family as ESSPRK($3,3,2$).
    %\davidtodo{This is a bit strange, since that notation was defined to refer to a single method.}
\end{theorem}
\begin{theorem}\label{thm:ESSPRK(4,3,2)}
	A family of optimal four-stage, effective order three SSP Runge--Kutta 
	methods of classical order two, with SSP coefficient $\sspcoef=2$ is given by
    \begin{displaymath}
    		\begin{split}
    			\bm{Y}_1 &= \bm{u}^n, \\
    			\bm{Y}_2 &= \bm{u}^n + \frac{1}{2}\Dt\bm{F}(\bm{Y}_1), \\
    			\bm{Y}_3 &= \bm{u}^n + \frac{1}{2}\Dt\bm{F}(\bm{Y}_1) + \frac{1}{2}\Dt\bm{F}(\bm{Y}_2), \\
    			\bm{Y}_4 &= \bm{u}^n + \gamma\Dt\bm{F}(\bm{Y}_1) + \gamma\Dt\bm{F}(\bm{Y}_2) + + \gamma\Dt\bm{F}(\bm{Y}_3), \\
    			\bm{u}^{n+1} &= \bm{u}^n + \frac{8\gamma-1}{12\gamma}\Dt\bm{F}(\bm{Y}_1) + \frac{1}{6}\Dt\bm{F}(\bm{Y}_2) + \frac{1}{6}\Dt\bm{F}(\bm{Y}_3) + \frac{1}{12\gamma}\Dt\bm{F}(\bm{Y}_4),
        \end{split}
    \end{displaymath}
    where $ 1/6 \leq \gamma \leq 1/2 $ is a free parameter.
    %We refer to the above family as ESSPRK($4,3,2$).
\end{theorem}
\begin{proof}
	In either theorem, feasibility can be verified by direct calculation of the 
	conditions in problem~\eqref{eq:SSP_opt}. Optimality follows because 
	$\sspcoef=\clin$.
\end{proof}

Theorem~\ref{thm:ESSPRK(3,3,2)} gives a \emph{family} of three-stage 
methods. 
The particular value of $\gamma = 1/4$ corresponds to the classical
Shu--Osher SSPRK($3,3$) method \cite{Gottlieb/Shu:1998}.
Similarly, in Theorem~\ref{thm:ESSPRK(4,3,2)} the particular value of 
$\gamma = 1/6$ corresponds to the usual SSPRK($4,3$) method.
It seems possible that for each number of stages, the 
ESSPRK($s, 3, 2$) methods may form a family in which an optimal 
SSPRK($s$, $3$) method is a particular member. 

\subsubsection{Effective order four methods}\label{subsubsec:4th_ESSPRK}
For effective order four, the ESSPRK($s,4,p$) methods can have
classical order $p=2$ or $3$.
%These methods have an SSP coefficient $\sspcoef$ which is at least as
%large as SSPRK($s, 4$) (in fact, they are all larger except in the
%case where $s =10$ where they are equal).\davidtodo{See my comment above.}
In either case, for stages $7 \le s \le 11$ the methods found are
optimal because the SSP coefficient attains the upper bound of
$\clin$.
For fewer stages, the ESSPRK methods still have SSP coefficients up to
30\% larger than that of explicit SSPRK($s,q$) methods.
% the SSP coefficient of ESSPRK($s,4,2$) and
%ESSPRK($s,4,3$) are the same and both achieve the linear bound
%$\clin$.
%ESSPRK($6,4,2$) is also optimal.
In the particular case of four-stage methods we have the following:% result:
\begin{remark}
	In contrast with the non-existence of an SSPRK(4,\,4) method 
	\cite{Gottlieb/Shu:1998,Ruuth2002}, 
	we are able to find ESSPRK(4,\,4,\,2) and ESSPRK(4,\,4,\,3) methods.
	The coefficients of these methods are found in
	Tables~\ref{tab:ESSPRK(4,4,2)_scheme}
	and~\ref{tab:ESSPRK(4,4,3)_scheme}.
\end{remark}
% Inside italics, notation looked funny with upright numerals, so
% I removed math mode (and added "\," for some extra space)

Similarly to \cite[Section~6.2.2]{Gottlieb2011a} we are able to find a family of 
methods with effective order four, for which $\ceff$ asymptomatically approaches 
the value one.
The family consists of second order methods with $n^1+1$ stages and 
SSP coefficient $n^2-n$.
It is convenient to express the coefficients in the Shu-Osher form,
because of the sparsity of the matrices $\alpha$ and $\beta$.
For $n \geq 3$ the non-zero elements are given by
\begin{align*}
	\alpha_{n^2-2n+4,(n-2)^2} &= \frac{n^2-1 \pm \sqrt{n^3-3n^2+n+1}}{4n^2-6n+2} \\
	\alpha_{n^2+1,1} &= \frac{2}{(n^2+1)\bigl((n-1)^2+1\bigr)} \\
	\alpha_{n^2+1,n^2-2n+2} &= 1 - \alpha_{n^2+1,1} - \frac{n(n-1)^2}{(2n-1)(n^2+1)(1-\alpha_{n^2-2n+4,(n-2)^2})} \\\\
	\alpha_{i+1,i} &= \begin{cases} 
								1 - \alpha_{i+1,(n-2)^2}, & i = n^2-2n+3 \\\\
								1 - \alpha_{i+1,1} - \alpha_{i+1,n^2-2n+2}, & i = n^2+1 \\\\
								1,  &\mbox{otherwise,}
							\end{cases}
\end{align*}
where $ 1 \leq i \leq n^2+1$ and 
\begin{align*}
	\beta_{i,j} & = \begin{cases} 
								0,  &i = n^2+2, j = 1 \\\\
								\dfrac{\alpha_{i,j}}{n^2-n}, & \mbox{otherwise,}
						\end{cases}
\end{align*}
for $ 1 \leq i \leq n^2+2$ and $ 1 \leq j \leq n^2+1$.

\subsection{Starting and stopping methods}\label{subsec:starting_stopping}
Having constructed an ESSPRK($s,q,p$) scheme that can be used as the main 
method $M$, we want to find perturbation methods $S$ and $S^{-1}$ such that the 
Runge--Kutta scheme $S^{-1}MS$ attains classical order $q$, equal to the 
effective order of method $M$.
We also want the resulting overall process to be SSP.
%If method $S$ advances the solution in time, then $S^{-1}$ must evolve the
%solution backward in time, which may be undesirable in some cases.
%For this reason, starting and stopping methods $S, S^{-1}$ are usually
%designed to perturb the solution but not advance in time.
%In any case, the resulting
However at least one of the $S$ and $S^{-1}$ methods is not SSP:
if $\beta_1 = 0$ then $\sum_i b_i = 0$ implies the presence of at
least one negative weight and thus neither scheme can be SSP.
Even if we consider methods with $\beta_1 \neq 0$, one of $S$ or
$S^{-1}$ must step backwards and thus that method cannot be SSP
(unless we consider the downwind operator
\cite{Ruuth2004,Gottlieb/Ruuth:SSPfastdownwind,Ketcheson:2011:downwind}).

In order to overcome this problem and achieve ``bona fide'' effective order 
SSPRK methods we need to choose different starting and stopping methods. 
We consider methods $R$ and $T$ which each take a positive step such that 
$R \equival{q} MS$ and $T \equival{q} S^{-1}M$.
That is, the order conditions of $R$ and $T$ must match those of
$MS$ and $S^{-1}M$, respectively, up to order $q$.
This gives a new $TM^{n-2}R$ scheme which is equivalent up to order $q$ 
to the $S^{-1}M^nS$ scheme and attains classical order $q$.
The starting and stopping procedures now each take a positive step forward 
in time.

To derive order conditions for the $R$ and $T$ methods, consider their
corresponding functions in group $G$ to be $\rho$ and $\tau$
respectively.
Then the equivalence is expressed as
\begin{equation} \label{eq:R_T_OCs}
    \rho(t) = (\beta\alpha)(t) \text{ and } \tau(t) = (\alpha\beta^{-1})(t), \quad \text{for all 
    trees $t$ with $r(t) \leq q$.}
\end{equation}
%and equivalently,
%\begin{equation} \label{eq:RMT_OCs}
%    \rho\alpha\tau(t) = E^3(t), \quad \text{for all trees $t$ with $r(t) \leq q$.}
%\end{equation}
Rewriting the second condition in \eqref{eq:R_T_OCs} as 
$(\tau\beta)(t) = \alpha(t)$, the order conditions for the starting and stopping 
methods can be determined and are given in Table~\ref{tab:rho_tau_OCs}.
These conditions could be constructed more generally but here we have
assumed $\beta_1=0$ (see Section~\ref{subsubsec:Main_starting_conditions}); this
will be sufficient for constructing SSP starting and stopping
conditions.


\begin{table}
	\centering
	\begin{tabular}{lcl}
		\hline
    		\multicolumn{1}{c}{$\rho(t) = (\beta\alpha)(t)$} & & \multicolumn{1}{c}{$\tau(t) = (\alpha\beta^{-1})(t)$} \\
    		\hline
    		 $\rho_1 = \alpha_1$ & & $\tau_1 = \alpha_1$ \\
    		$\rho_2 = \alpha_2 + \beta_2$ & & $\tau_2 = \alpha_2 - \beta_2$ \\
    		$\rho_3 = \alpha_3 + \beta_3$ & & $\tau_3 = \alpha_3 - 2\alpha_1\beta_2 - \beta_3$ \\
    		$\rho_4 = \alpha_4 + \alpha_1\beta_2 + \beta_4$ & & $\tau_4 = \alpha_4 - \alpha_1\beta_2 - \beta_4$ \\
                \hdashline[2pt/3pt] \\[-10pt]
		$\rho_5 = \alpha_5 + \beta_5$ & & $\tau_5 = \alpha_5 - 3\alpha_1^2\beta_2 - 3\alpha_1\beta_3 - \beta_5$ \\
		$\rho_6 = \alpha_6 + \alpha_2\beta_2 + \beta_6$ & & $\tau_6 = \alpha_6 - (\alpha_1^2 + \alpha_2 -\beta_2)\beta_2 -\alpha_1\beta_3 - \alpha_1\beta_4 - \beta_6$ \\
		$\rho_7 = \alpha_7 + \alpha_1\beta_3 + \beta_7$ & & $\tau_7 = \alpha_7 - 2\alpha_1\beta_4 - \alpha_1^2\beta_2 - \beta_7$ \\
		$\rho_8 = \alpha_8 + \alpha_1\beta_4 + \alpha_2\beta_2 + \beta_8$ & & $\tau_8 = \alpha_8 - \alpha_1\beta_4 - \alpha_2\beta_2 + \beta_2^2 -  \beta_8$
  	\end{tabular}
  	\caption{Order conditions on $\rho$ and $\tau$ up to effective order four for starting
  		and stopping methods $R$ and $T$, respectively.
  		The upper block represents the effective order three conditions.
         As in Tables~\ref{tab:effective_OCs_on_alpha} and \ref{tab:effective_OCs} we assume 
         $\beta_1 = 0$.}
  	\label{tab:rho_tau_OCs}
\end{table}

\subsubsection{Optimizing the starting and stopping methods}\label{subsubsec:opt_methods}
It turns out that the order conditions from \eqref{eq:R_T_OCs} do not
contradict the SSP requirements.
We can thus find methods $R$ and $T$ using the optimization procedure
described in Section~\ref{subsec:Optimal_SSPRK} with the order conditions 
given by Table~\ref{tab:rho_tau_OCs} for $\Phi(K)$ in \eqref{eq:SSP_opt}.

The values of $\alpha_i$ are determined by the main method $M$.
Also note that for effective order $q$, the algebraic expressions on
$\beta$ up to order $q-1$ are already found by the optimization procedure of 
the main method (see Table~\ref{tab:effective_OCs}). 
However, the values of the order $q$ elementary weights on $\beta$ are not 
known; these are $\beta_3$ and $\beta_4$ for effective order three and
$\beta_5$, $\beta_6$, $\beta_7$ and $\beta_8$ for effective order four.
From Table~\ref{tab:rho_tau_OCs}, we see that both the $R$ and $T$
methods depend on these parameters.
Our approach is to optimize for both methods at once: we solve a
modified version of the optimization problem \eqref{eq:SSP_opt} where
we simultaneously maximize both SSP coefficients subject to the
constraints given in \eqref{eq:R_T_OCs} and conditions on $\beta$ given by 
Table~\ref{tab:effective_OCs}. 
The unknown elementary weights on $\beta$ are used as free parameters.
In practice, we maximize the objective function $\min(r_1,r_2)$, where $r_1$ 
and $r_2$ are the radii of absolute monotonicity of the methods $R$ and $T$.

We were able to construct starting and stopping schemes for each main 
method, with an SSP coefficient at least as large as that of the main method.
This allows the usage of a uniform time-step $\Dt \leq \sspcoef\DtFE$, 
where $\sspcoef$ is the SSP coefficient of the main method.
The additional computational cost
of the starting and stopping methods is minimal:
for methods $R$ and $T$ associated with an $s$-stage main method,
at most $s + 1$ and $s$ stages, respectively, appear to be required.
%In some stages better results are obtained, for example if the main method is 
%ESSPRK($3,3,2$), then its relative starting and stopping methods have only three 
%stages.
%\begin{table}
%    \centering
%	\centering
%	\begin{tabular}{ccc}
%          $q$ & $p$ & number of stages each for $R$ and $T$\\
%          \hline
%          3 & 2 & s+1 ??? \\
%          4 & 2 & s+1??? \\
%          4 & 3 & s+2???
%  	\end{tabular}
%        \qquad
%    \begin{tabular}{|c|c|cccccccc|}
%        \hline
%        \multicolumn{2}{|c|}{\backslashbox{\hspace{1pt}\vspace{1pt}$p$}{\vspace{-5pt}$s$}} & $4$ & $5$ & $6$ & $7$ & $8$ & $9$ & $10$ & $11$ \\
%        \hline
%        \multirow{2}{*}{$2$} & $R$ & $5$ & $7$ & $7$ & $8$ & $9$ & $10$ & $11$ & $12$ \\
%        & $T$ & $5$ & $5$ & $6$ & $7$ & $8$ & $9$ & $10$ & $11$ \\
%        \hline
%        \multirow{2}{*}{$3$}& $R$ & $5$ & $6$ & $7$ & $8$ & $9$ & $10$ & $11$ & $12$ \\
%        & $T$ & $5$ & $5$ & $7$ & $8$ & $8$ & $10$ & $11$ & $12$ \\
%        \hline
%    \end{tabular}
%    \caption{Minimum number of stages required for the starting and
%          stopping method $R$ and $T$ for each ESSPRK($s,4,p$)
%          main method.
%          \textbf{TODO: double check these, then decide which table we want.}
%        }
%    \label{tab:RT_stages}
%\end{table}
Tables \ref{tab:ESSPRK(4,4,2)_scheme} and \ref{tab:ESSPRK(4,4,3)_scheme} 
show the coefficients of the effective SSP schemes in the case the main 
method is ESSPRK($4,4,2$) and ESSPRK($4,4,3$), respectively. 

It is important to note that in practice, if accurate values are needed at 
any time other than the final time, the computation must invoke the 
stopping method to obtain them.  Furthermore, changing step-size would require first applying the stopping method with the old step-size and then applying the starting method with the new step-size.

%, and then the starting method to start up again if the stepsize where to change.

\begin{table}
    \setlength{\tabcolsep}{2pt}
    %\centering
    \footnotesize
    \subfloat[Main method $M$, ESSPRK($4,4,2$) \label{ESSPRK(4,4,2)_scheme_a}]{
        \begin{tabular}{c | c c c c}
             $0$ & & & & \\
             $0.730429885783319$ & $0.730429885783319$ & & & \\
             $0.644964638145795$ & $0.251830917810810$ & $0.393133720334985$ & & \\
             $1.000000000000000$ & $0.141062771617064$ & $0.220213358584678$ & $0.638723869798257$ & \\
             \hline
             & $0.384422161080494$ & $0.261154113377550$ & $0.127250689937518$ & $0.227173035604438$
        \end{tabular}
    }\\
    \subfloat[Starting method $R$ \label{ESSPRK(4,4,2)_scheme_b}]{
        \begin{tabular}{c | c c c c c}
			$0$ & & & & & \\
			$0.545722177514735$ & $0.545722177514735$ & & & & \\
			$0.842931687441527$ & $0.366499989048164$ & $0.476431698393363$ & & & \\
			$0.574760809487828$ & $0.135697968350722$ & $0.176400587890242$ & $0.262662253246864$ & & \\
			$0.980872743236632$ & $0.103648417776838$ & $0.134737771331049$ & $0.200625899485633$ & $0.541860654643112$ & \\		
             \hline
             & $0.233699169638954$ & $0.294263351266422$ & $0.065226988215286$ & $0.176168374199685$ & $0.230642116679654$ 
        \end{tabular}        
    }\\
    \subfloat[Stopping method $T$ \label{ESSPRK(4,4,2)_scheme_c}]{
        \begin{tabular}{c | c c c c}
			$0$ & & & & \\
			$0.509877496215340$ & $0.509877496215340$ & & & \\
			$0.435774135529007$ & $0.182230305923759$ & $0.253543829605247$ & & \\
			$0.933203341300203$ & $0.148498121305090$ & $0.206610981494095$ & $0.578094238501017$ & \\
             \hline            
             & $0.307865440399752$ & $0.171863794704750$ & $0.233603236964822$ & $0.286667527930676$
        \end{tabular}
    }
    \caption{ESSPRK(4,4,2): an effective order four SSPRK method with
      four stages and classical order two with its associated starting
      and stopping methods.}
    \label{tab:ESSPRK(4,4,2)_scheme}
\end{table}

\begin{table}
    \setlength{\tabcolsep}{2pt}
    %\centering
    \footnotesize
    \subfloat[Main method $M$, ESSPRK($4,4,3$) \label{ESSPRK(4,4,3)_scheme_a}]{
        \begin{tabular}{c | c c c c}
             $0$ & & & & \\
             $0.601245068769724$ & $0.601245068769724$ & & & \\
             $0.436888719886063$ & $0.139346829159954$ & $0.297541890726109$ & & \\
             $0.747760163757110$ & $0.060555450075478$ & $0.129301708677891$ & $0.557903005003740$ & \\
             \hline
             & $0.220532078662434$ & $0.180572397883936$ & $0.181420582644840$ & $0.417474940808790$
        \end{tabular}  
    }\\
    \subfloat[Starting method $R$ \label{ESSPRK(4,4,3)_scheme_b}]{
        \begin{tabular}{c | c c c c c}
			$0$ &  & & & & \\
			$0.438463764036947$ & $0.438463764036947$ & & & & \\
			$0.639336395725557$ & $0.213665532574654$ & $0.425670863150903$ & & & \\
			$0.434353425654020$ &$0.061345094040860$ & $0.122213530726218$ & $0.250794800886942$ & & \\
			$0.843416464962307$ & $0.039559973266996$ & $0.078812561688700$ & $0.161731525131914$ & $0.563312404874697$ & \\			
             \hline
             & $0.154373542967849$ & $0.307547588471376$ & $0.054439037790856$ & $0.189611674483496$ & $0.294028156286422$
        \end{tabular}
    }\\
    \subfloat[Stopping method $T$ \label{ESSPRK(4,4,3)_scheme_c}]{
        \begin{tabular}{c | c c c c}
			$0$ & & & & \\
			$0.556337718891090$ & $0.556337718891090$ & & & \\
			$0.428870688216872$ & $0.166867537553458$ & $0.262003150663414$ & & \\
			$0.815008947642716$ & $0.104422177204659$ & $0.163956032598547$ & $0.546630737839510$ & \\		
             \hline
             & $0.203508169408374$ & $0.096469758967330$ & $0.321630956102914$ & $0.378391115521382$
        \end{tabular}
    }
    \caption{ESSPRK(4,4,3): an effective order four SSPRK method with
      four stages and classical order three with its associated starting
      and stopping methods.}
    \label{tab:ESSPRK(4,4,3)_scheme}
\end{table}
