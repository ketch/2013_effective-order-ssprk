\section{Optimal effective order SSP Runge--Kutta schemes}\label{sec:optimal_ESSPRK}
In this section, we use the SSP theory and Butcher's theory of effective
order (Sections \ref{sec:SSP} and \ref{sec:Algebraic_RK}) to find
optimal explicit SSP Runge--Kutta schemes with prescribed effective
order and classical order.
Due to Corollary~\ref{cor:no_SSP_5}, there are no SSPRK methods of
effective order five, and therefore we need only consider methods with
effective order up to four.

Recall from Section~\ref{sec:Algebraic_RK} that the methods of
effective order involve a main method $M$ as well as starting and
stopping methods $S$ and $S^{-1}$.
It turns out that methods $S$ and $S^{-1}$ are not generally SSP, but
we show below in Section~\ref{subsec:starting_stopping} how to modify
the procedure to fix this issue.

We denote by ESSPRK($s,q,p$) an $s$--stage explicit SSP Runge--Kutta 
method of effective order $q$ and classical order $p$.
Also we write SSPRK($s,q$) for a  $s$-stage explicit SSP Runge--Kutta 
method of order $q$.

\subsection{The main method}\label{subsec:main_method}

Our search for methods of effective order can be carried out in two
steps, first searching for optimal main methods $M$ and then for
possible corresponding methods $S$ and $S^{-1}$.
For a given number of stages, effective order, and classical order,
our aim is thus to find an optimal main method, that is one with the 
largest possible SSP coefficient $\sspcoef$ for nonlinear problems.

To find a method ESSPRK($s,q,p$) with Butcher tableau $(A, \bm{b},
\bm{c})$, we consider the optimization problem~\eqref{eq:SSP_opt} 
now with $\Phi_{q,p}(K)$ representing the conditions for effective order
$q$ and classical order $p$ (as per Table~\ref{tab:effective_OCs}).
The methods are found through numerical search, using 
\textsc{MATLAB}'s optimization toolbox on the above constrained 
optimization problem. 
Specifically, we use \texttt{fmincon} with a sequential quadratic 
programming approach \cite{Ketcheson2008}.
This process does not guarantee a global minimizer, so many searches 
from random initial guesses are performed to ensure the method with 
the largest possible SSP coefficient is found.

\subsubsection{Optimal SSP coefficients}\label{subsubsec:optimal_SSP_coeff}
The effective SSP coefficients for up to eleven stages are shown in 
Table~\ref{tab:eff_SSP_coeff}. 
Similarly to SSPRK($s,q$) methods, all ESSPRK($s,q,p$) methods 
have $\ceff < 1$. 
Since the effective order $q$ SSPRK methods (accompanied with relevant 
starting and stopping methods) are used in a scheme that attains classical 
order equal to $q$, it is reasonable to compare them to the SSPRK($s,q$) 
methods.
We then have the following results:
\begin{result}
	ESSPRK($s,q,p$) methods have SSP coefficients $\sspcoef$ greater than 
	or equal to that of SSPRK($s,q$) methods.
\end{result}
\begin{result}
	Many of the ESSPRK($s,q,p$) methods (and all of those with $s \ge 7$ 
	stages) we have found by numerical search, have SSP coefficients 
	$\sspcoef$ which achieve the linear bound. 
	This shows they are optimal.  
\end{result}

\begin{table}
    \centering
    \begin{tabular}{|c|c|ccccccccccc|}
        \hline
        \multicolumn{2}{|c|}{\backslashbox{\hspace{2pt}\vspace{1pt}$q\,,\,p$}{\vspace{-5.5pt}$s$}} & $1$ & $2$ & $3$ & $4$ & $5$ & $6$ & $7$ & $8$ & $9$ & $10$ & $11$ \\
        \hline
        $q = 3$ & $p = 2$ & $-$ & $-$ & $\bf 0.33$ & $\bf 0.50$ & $\bf 0.53$ & $\bf 0.59$ & $\bf 0.61$ & $\bf 0.64$ & $\bf 0.67$ & $\bf 0.68$ & $\bf 0.69$\\
        \hline
        $q = 4$ & $p = 2$ & $-$ & $-$ & $-$ & $0.22$ & $0.39$ & $\bf 0.44$ & $\bf 0.50$ & $\bf 0.54$ & $\bf 0.57$ & $\bf 0.60$ & $\bf 0.62$ \\
        \hline
        $q = 4$ & $p = 3$ & $-$ & $-$ & $-$ & $0.19$ & $0.37$ & $0.43$ & $\bf 0.50$ & $\bf 0.54$ & $\bf 0.57$ & $\bf 0.60$ & $\bf 0.62$ \\
        \hline
    \end{tabular}
    \caption{Effective SSP coefficients $ \ceff = \sspcoef/s$ of best known explicit effective 
    		order ESSPRK($s,q,p$) methods. 
    		Entries in bold achieve the linear bound on the SSP coefficient and are therefore optimal. 
    		If no positive $\sspcoef$ can be found, we use ``$-$'' to indicate non-existence. 
    		The optimal fourth-order linear bounds for $s=4$ and $s=5$ are $0.25$ and $0.40$ 
    		respectively.}
    \label{tab:eff_SSP_coeff}
\end{table}

\subsubsection{Effective order three methods}\label{subsubsec:3rd_ESSPRK}
Table~\ref{tab:eff_SSP_coeff} shows that ESSPRK($s,3,2$) are 
optimal because they match the linear bounds. 
Note however that these bounds are also attained by SSPRK($s,3$) methods 
up to 11 stages. 
In the cases of three and four stages, we were able to determine closed-form 
families of solutions.
\begin{theorem}\label{thm:ESSPRK(3,3,2)}
	An optimal family of three-stage, third effective order SSP Runge--Kutta 
	methods of classical order two, with SSP coefficient $\sspcoef=1$ is given by
    \begin{displaymath}
    		\begin{split}
    			\bm{Y}_1 &= \bm{u}^n \\
    			\bm{Y}_2 &= \bm{u}^n + \Dt\bm{F}(\bm{Y}_1) \\
    			\bm{Y}_3 &= \bm{u}^n + \gamma\Dt\bm{F}(\bm{Y}_1) + \gamma\Dt\bm{F}(\bm{Y}_2) \\
    			\bm{u}^{n+1} &= \bm{u}^n + \frac{5\gamma-1}{6\gamma}\Dt\bm{F}(\bm{Y}_1) + \frac{1}{6}\Dt\bm{F}(\bm{Y}_2) + \frac{1}{6\gamma}\Dt\bm{F}(\bm{Y}_3),
        \end{split}
    \end{displaymath}
    where $1/4 \leq \gamma \leq 1$ is a free parameter. We denote the above family as 
    ESSPRK($3,3,2$).
\end{theorem}
\begin{theorem}\label{thm:ESSPRK(4,3,2)}
	An optimal family of four-stage, third effective order SSP Runge--Kutta 
	methods of classical order two, with SSP coefficient $\sspcoef=2$ is given by
    \begin{displaymath}
    		\begin{split}
    			\bm{Y}_1 &= \bm{u}^n \\
    			\bm{Y}_2 &= \bm{u}^n + \frac{1}{2}\Dt\bm{F}(\bm{Y}_1) \\
    			\bm{Y}_3 &= \bm{u}^n + \frac{1}{2}\Dt\bm{F}(\bm{Y}_1) + \frac{1}{2}\Dt\bm{F}(\bm{Y}_2) \\
    			\bm{Y}_4 &= \bm{u}^n + \gamma\Dt\bm{F}(\bm{Y}_1) + \gamma\Dt\bm{F}(\bm{Y}_2) + + \gamma\Dt\bm{F}(\bm{Y}_3) \\
    			\bm{u}^{n+1} &= \bm{u}^n + \frac{8\gamma-1}{12\gamma}\Dt\bm{F}(\bm{Y}_1) + \frac{1}{6}\Dt\bm{F}(\bm{Y}_2) + \frac{1}{6}\Dt\bm{F}(\bm{Y}_3) + \frac{1}{12\gamma}\Dt\bm{F}(\bm{Y}_4),
        \end{split}
    \end{displaymath}
    where $ 1/6 \leq \gamma \leq 1/2 $ is a free parameter. We denote the 
    above family as ESSPRK($4,3,2$).
\end{theorem}
\begin{proof}
	In either theorem, feasibility can be verified by direct calculation of the 
	conditions in problem~\eqref{eq:SSP_opt}. Optimality follows because 
	the families achieve the appropriate linear bound.
\end{proof}

Theorem~\ref{thm:ESSPRK(3,3,2)} gives a \emph{family} of three-stage 
methods. 
The particular value of $\gamma = 1/4$ corresponds to the classical
Shu--Osher SSPRK($3,3$) method \cite{Gottlieb/Shu:1998}.
Similarly, in Theorem~\ref{thm:ESSPRK(4,3,2)} the particular value of 
$\gamma = 1/6$ corresponds to the usual SSPRK($4,3$) method.
It seems possible that for each number of stages, the 
ESSPRK($s, 3, 2$) methods may form a family in which an optimal 
SSPRK($s$, $3$) method is a particular member. 

\subsubsection{Effective order four methods}\label{subsubsec:4th_ESSPRK}
For effective order four, the ESSPRK($s,4,p$) methods can have
classical order $p=2$ or $3$.
These methods have an SSP coefficient $\sspcoef$ which is at least as
large as SSPRK($s, 4$) (in fact, they are all larger except in the
case where $s =10$ where they are equal).
For stages $s \ge 7$ the SSP coefficient of ESSPRK($s,4,2$) and
ESSPRK($s,4,3$) are the same and both achieve the linear bound
$\clin$.
ESSPRK($6,4,2$) is also optimal.
In the particular case of four-stage methods we have the following result:
\begin{result}
	In contrast with the non-existence of an SSPRK(4,\,4) method 
	\cite{Gottlieb/Shu:1998,Ruuth2002}, 
	we were are able to find ESSPRK(4,\,4,\,2) and ESSPRK(4,\,4,\,3) methods.
	The coefficients of these methods are found in
	Tables~\ref{tab:ESSPRK(4,4,2)_scheme}
	and~\ref{tab:ESSPRK(4,4,3)_scheme}.
\end{result}
% Inside italics, notation looked funny with upright numerals, so
% I removed math mode (and added "\," for some extra space)
% GOOD!

\subsection{Starting and stopping methods}\label{subsec:starting_stopping}
Having constructed an ESSPRK($s,q,p$) scheme that can be used as the main 
method $M$, we want to find perturbation $S$ and $S^{-1}$ such that the 
Runge--Kutta scheme $S^{-1}MS$ attains classical order $q$, equal to the 
effective order of method $M$.
We also want the resulting overall process to be SSP.
If method $S$ advances the solution in time, then $S^{-1}$ must evolve the 
solution backward in time, which may be undesirable in some cases. 
For this reason, starting and stopping methods $S, S^{-1}$ are usually 
designed to perturb the solution but not advance in time. 
A drawback of this approach is that the resulting $S$ (and $S^{-1}$) scheme 
is not SSP:
the condition $\beta_1 = 0$ (i.e., $\sum_i b_i = 0$) implies negative
weights and thus the scheme cannot be SSP.
The overall $S^{-1}M^nS$ method will therefore also not be SSP.

In order to overcome this problem and achieve ``bona fide'' effective order 
SSPRK methods we need to choose different starting and stopping methods. 
We consider methods $R$ and $T$ which each take a positive step such that 
$R \underaccent{p}{\equiv} MS$ and $T \underaccent{p}{\equiv}S^{-1}M$.
That is, the order conditions of $R$ and $T$ must match those of
$MS$ and $S^{-1}M$, respectively, up to order $q$.
This gives a new $TM^{n-2}R$ method which is equivalent up to order $q$ 
to the $S^{-1}M^nS$ method and attains classical order $q$.
The starting and stopping procedures now each take a positive step forward 
in time.

To derive order conditions for the $R$ and $T$ methods, consider their
corresponding functions in group $G$ to be $\rho$ and $\tau$
respectively.
Then the equivalence is expressed as
\begin{equation} \label{eq:R_T_OCs}
    \rho(t) = (\beta\alpha)(t) \text{ and } \tau(t) = (\alpha\beta^{-1})(t), \quad \text{for all 
    trees $t$ with $r(t) \leq q$.}
\end{equation}
%and equivalently,
%\begin{equation} \label{eq:RMT_OCs}
%    \rho\alpha\tau(t) = E^3(t), \quad \text{for all trees $t$ with $r(t) \leq q$.}
%\end{equation}
Rewriting the second condition in \eqref{eq:R_T_OCs} as 
$(\tau\beta)(t) = \alpha(t)$, the order conditions for the starting and stopping 
methods can be determined and are given in Table~\ref{tab:rho_tau_OCs}.

\begin{table}
	\centering
	\begin{tabular}{lcl}
		\hline
    		\multicolumn{1}{c}{$\rho(t) = (\beta\alpha)(t)$} & & \multicolumn{1}{c}{$(\tau\beta)(t) = \alpha(t)$} \\
    		\hline
    		 $\rho_1 = \alpha_1$ & & $\tau_1 = \alpha_1$ \\
    		$\rho_2 = \alpha_2 + \beta_2$ & & $\tau_2 = \alpha_2 - \beta_2$ \\
    		$\rho_3 = \alpha_3 + \beta_3$ & & $\tau_3 = \alpha_3 - 2\alpha_1\beta_2 - \beta_3$ \\
    		$\rho_4 = \alpha_4 + \alpha_1\beta_2 + \beta_4$ & & $\tau_4 = \alpha_4 - \alpha_1\beta_2 - \beta_4$ \\
                \hdashline[2pt/3pt]
		$\rho_5 = \alpha_5 + \beta_5$ & & $\tau_5 = \alpha_5 - 3\alpha_1^2\beta_2 - 3\alpha_1\beta_3 - \beta_5$ \\
		$\rho_6 = \alpha_6 + \alpha_2\beta_2 + \beta_6$ & & $\tau_6 = \alpha_6 - (\alpha_1^2 + \alpha_2 -\beta_2)\beta_2 -\alpha_1\beta_3 - \alpha_1\beta_4 - \beta_6$ \\
		$\rho_7 = \alpha_7 + \alpha_1\beta_3 + \beta_7$ & & $\tau_7 = \alpha_7 - 2\alpha_1\beta_4 - \alpha_1^2\beta_2 - \beta_7$ \\
		$\rho_8 = \alpha_8 + \alpha_1\beta_4 + \alpha_2\beta_2 + \beta_8$ & & $\tau_8 = \alpha_8 - \alpha_1\beta_4 - \alpha_2\beta_2 + \beta_2^2 -  \beta_8$
  	\end{tabular}
  	\caption{Order conditions on $\rho$ and $\tau$ up to fourth effective order for starting 
  	and stopping methods $R$ and $T$, respectively.
        The upper block represents the third 
  	effective order conditions.}
  	\label{tab:rho_tau_OCs}
\end{table}

\subsubsection{Optimizing the staring and stopping methods}\label{subsubsec:opt_methods}
It turns out that the order conditions from \eqref{eq:R_T_OCs} do not
contradict the SSP requirements.
We can thus FIND METHODS $R$ and $T$ using the optimization procedure 
described in Section~\ref{subsec:Optimal_SSPRK} with the order conditions 
given by Table~\ref{tab:rho_tau_OCs} instead of the $\Phi_{p,q}$
in \eqref{eq:SSP_opt}.

For a given main method $M$, the values of $\alpha_i$ are already known.
Also note that for either case of effective order $q$, the algebraic expressions on 
$\beta$ up to order $q-1$ are already found by the optimization procedure of 
the main method (see Table~\ref{tab:effective_OCs}). 
However, the values of the order $q$ elementary weights on $\beta$ are not 
known; these are $\beta_3$ and $\beta_4$ for effective order three and
$\beta_5$, $\beta_6$, $\beta_7$ and $\beta_8$ for effective order four.
From Table~\ref{tab:rho_tau_OCs}, we see that both the $R$ and $T$
methods depend on these parameters.
Our approach is to optimize for both methods at once: we solve a
modified version of the optimization problem \eqref{eq:SSP_opt} where
we simultaneously maximize both SSP coefficients subject to the
constraints given in \eqref{eq:R_T_OCs} and conditions on $\beta$ given by 
Table~\ref{tab:effective_OCs}. 
The unknown elementary weights on $\beta$ are declared as free parameters.
\colintodo{Sort out whether we need this $\beta\beta^{-1}$ condition.}
\yiannistodo{We don't. \eqref{eq:R_T_OCs} implies that and also is derived by definition of inverse.}
In practice, we maximize the objective function $\min(r_1,r_2)$, where $r_1$ 
and $r_2$ are the absolute radii of monotonicity of the two Runge--Kutta 
methods.

\subsection{Resulting schemes}\label{subsec:resulting_schemes}
We were able to construct starting and stopping schemes for each main 
method, with an SSP coefficient at least as large as that of the main method.
For each order, effective order and number of stages the computational cost
of the starting and stopping methods is kept low. 
In particular, methods $R$ and $T$ related to an $s$-stage main method, only 
require to have $s + 1$ and $s$ stages. 
In some stages better results are obtained, for example if the main method is 
ESSPRK($3,3,2$), then its relative starting and stopping methods have only three 
stages.
For a main method of classical order two, the starting and stopping schemes 
themselves are first order accurate, whereas for classical order three they 
achieve second order accuracy. 

%\begin{table}
%    \centering
%	\centering
%	\begin{tabular}{ccc}
%          $q$ & $p$ & number of stages each for $R$ and $T$\\
%          \hline
%          3 & 2 & s+1 ??? \\
%          4 & 2 & s+1??? \\
%          4 & 3 & s+2???
%  	\end{tabular}
%        \qquad
%    \begin{tabular}{|c|c|cccccccc|}
%        \hline
%        \multicolumn{2}{|c|}{\backslashbox{\hspace{1pt}\vspace{1pt}$p$}{\vspace{-5pt}$s$}} & $4$ & $5$ & $6$ & $7$ & $8$ & $9$ & $10$ & $11$ \\
%        \hline
%        \multirow{2}{*}{$2$} & $R$ & $5$ & $7$ & $7$ & $8$ & $9$ & $10$ & $11$ & $12$ \\
%        & $T$ & $5$ & $5$ & $6$ & $7$ & $8$ & $9$ & $10$ & $11$ \\
%        \hline
%        \multirow{2}{*}{$3$}& $R$ & $5$ & $6$ & $7$ & $8$ & $9$ & $10$ & $11$ & $12$ \\
%        & $T$ & $5$ & $5$ & $7$ & $8$ & $8$ & $10$ & $11$ & $12$ \\
%        \hline
%    \end{tabular}
%    \caption{Minimum number of stages required for the starting and
%          stopping method $R$ and $T$ for each ESSPRK($s,4,p$)
%          main method.
%          \textbf{TODO: double check these, then decide which table we want.}
%        }
%    \label{tab:RT_stages}
%\end{table}

Tables \ref{tab:ESSPRK(4,4,2)_scheme} and \ref{tab:ESSPRK(4,4,3)_scheme} 
show the coefficients of the effective SSP schemes in the case the main 
method is ESSPRK($4,4,2$) and ESSPRK($4,4,3$), respectively. 

It is important to note that in practice, if accurate values are needed at 
any time other than the final time, the computation must invoke the 
stopping method, and then the starting method to start up again.

\begin{table}
    \setlength{\tabcolsep}{2pt}
    \centering
    \scriptsize
    \subfloat[Main method $M$, ESSPRK($4,4,2$) \label{ESSPRK(4,4,2)_scheme_a}]{
        \begin{tabular}{c | c c c c}
             $0$ & & & & \\
             $0.730429885784024$ & $0.730429885784024$ & & & \\
             $0.644963741710099$ & $0.251830567791499$ & $0.393133173918601$ & & \\
             $1.000000000000000$ & $0.141062646387053$ & $0.220213163087548$ & $0.638724190525399$ & \\
             \hline
             & $0.384422058644697$ & $0.261153837271738$ & $0.127250866803383$ & $0.227173237280182$
        \end{tabular}
    }\\
    \subfloat[Starting method $R$ \label{ESSPRK(4,4,2)_scheme_b}]{
        \begin{tabular}{c | c c c c c}
			$0$ & $0$ & & & \\
			$0.373522916411400$ & $0.373522916411400$ & & & & \\
			$0.427950722860423$ & $0.131590527793077$ & $0.296360195067346$ & & & \\
			$0.638639971034253$ & $0.066215361487869$ & $0.149126215815910$ & $0.423298393730474$ & & \\
			$1.000000000000000$ & $0.059378615605483$ & $0.099226317694128$ & $0.281656318212781$ & $0.559738748487608$ & \\		
             \hline
             & $0.167820631351745$ & $0.266506504585377$ & $0.094688801056065$ & $0.188176112416812$ & $0.282807950590000$ 
        \end{tabular}
    }\\
    \subfloat[Stopping method $T$ \label{ESSPRK(4,4,2)_scheme_c}]{
        \begin{tabular}{c | c c c c c}
			$0$ & & & & & \\
			$0.502859501089623$ & $0.502859501089623$ & & & & \\
			$0.932299154557159$ & $0.466149577278579$ & $0.466149577278579$ & & & \\
			$0.592066180887698$ & $0.291363412015344$ & $0.144655475088422$ & $0.156047293783932$ & & \\
			$0.820219483624087$ & $0.389493128711985$ & $0.077538391027073$ & $0.083644646541993$ & $0.269543317343036$ & \\
             \hline            
             & $0.357516861171415$ & $0.120524890425247$ & $0.051000692115440$ & $0.164348781516878$ & $0.306608774771021$ 
        \end{tabular}
    }
    \caption{ESSPRK(4,4,2): an effective order four SSPRK method with
      four stages and classical order two with its associated starting
      and stopping schemes.}
    \label{tab:ESSPRK(4,4,2)_scheme}
\end{table}

\begin{table}
    \setlength{\tabcolsep}{2pt}
    \centering
    \scriptsize
    \subfloat[Main method $M$, ESSPRK($4,4,3$) \label{ESSPRK(4,4,3)_scheme_a}]{
        \begin{tabular}{c | c c c c}
             $0$ & & & & \\
             $0.601245068769723$ & $0.601245068769723$ & & & \\
             $0.436888719184949$ & $0.139346828936332$ & $0.297541890248617$ & & \\
             $0.747760163759696$ & $0.060555450003183$ & $0.129301708523521$ & $0.557903005232992$ & \\
             \hline
             & $0.220532078494543$ & $0.180572397262273$ & $0.181420582935982$ & $0.417474941307203$
        \end{tabular}
    }\\
    \subfloat[Starting method $R$ \label{ESSPRK(4,4,3)_scheme_b}]{
        \begin{tabular}{c | c c c c c}
			$0$ & $0$ & & & & \\
			$0.233583075787679$ & $0.233583075787679$ & & & & \\
			$0.515691493839167$ & $0.113222521489861$ & $0.402468972349307$ & & & \\
			$0.531082309996490$ &$0.044673367659019$ & $0.158799187092104$ & $0.327609755245368$ & & \\
			$0.781350492733201$ & $0.125069004030275$ & $0.133636207609711$ & $0.147871644558862$ & $0.374773636534354$ & \\			
             \hline
             & $0.154940460576884$ & $0.122447291145256$ & $0.196845001120610$ & $0.163510218326813$ & $0.362257028830438$
        \end{tabular}
    }\\
    \subfloat[Stopping method $T$ \label{ESSPRK(4,4,3)_scheme_c}]{
        \begin{tabular}{c | c c c c c c c}
			$0$ & & & & & \\
			$0.439651096124100$ & $0.439651096124100$ & & & & \\
			$0.703468818352023$ & $0.297025122670221$ & $0.406443695681802$ & & & \\
			$0.428982979460892$ & $0.097632937435411$ & $0.133599109579618$ & $0.197750932445863$ & & \\
			$0.874787742275635$ & $0.082872736093259$ & $0.113401522491430$ & $0.167854837386445$ & $0.510658646304502$ & \\			
             \hline
             & $0.182401342203530$ & $0.219128338221130$ & $0.078473760154594$ & $0.238737856798909$ & $0.281258702621837$
        \end{tabular}
    }
    \caption{ESSPRK(4,4,3): an effective order four SSPRK method with
      four stages and classical order three with its associated starting
      and stopping schemes.}
    \label{tab:ESSPRK(4,4,3)_scheme}
\end{table}
