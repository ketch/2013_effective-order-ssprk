\section{Introduction}\label{sec:Intro}
Solutions of nonlinear hyperbolic partial differential equations (PDEs) may 
contain discontinuities even when the initial conditions are smooth.
The challenge for the numerical solution of these systems is twofold. 
It is desirable that the approximation be of high accuracy in regions where 
the solution is smooth and that the discontinuities be captured without 
exhibiting any oscillations or overshoots. 

There has been considerable effort to develop, for these and other
problem classes, numerical methods which are \emph{strongly stable}.
Many of these numerical methods are based on a method-of-lines approach 
where the problem is first discretized in space to yield a system of ODEs. 
The spatial discretization is often chosen to ensure certain strong stability 
properties of the original PDE problem (e.g., max-norm monotonicity, 
total variation boundedness, positivity, etc) are preserved
\emph{when coupled with first-order Forward Euler time integration}.
Strong stability preserving (SSP) time discretizations (formerly TVD
discretizations \cite{Gottlieb/Shu:1998}) are high-order time
discretizations that guarantee the same stability preservation, with a
possibly different step-size restriction.

We examine the SSP properties of explicit Runge--Kutta methods of 
\emph{effective order}. 
Effective order methods use special starting and stopping procedures
in such a way that the method can achieve an order of
accuracy higher than its classical design order.
This allows the construction of high-order SSP Runge--Kutta schemes by
using low-order SSP Runge--Kutta methods.

Explicit SSP Runge--Kutta methods of classical order four require at 
least five stages \cite{Gottlieb/Shu:1998}. 
In contrast, we construct explicit SSP Runge--Kutta methods
of effective order four with only four stages.
Following this result, we had hoped to overcome the fifth-order
barrier for explicit SSP Runge--Kutta methods \cite{Ruuth2002}; instead we
prove that the barrier also holds for SSP methods of effective
order five.
%This is established later as a corollary of a more general theorem.
Most of the methods we find are optimal, as they achieve a certain theoretical
upper bound on the SSP coefficients that is obtained by considering only
linear problems \cite{Kraaijevanger1986}.

The rest of the paper is organized as follows.
Section~\ref{sec:SSP} reviews Runge--Kutta methods and the concept of
strong stability preserving methods.
Section~\ref{sec:Algebraic_RK} presents a brief overview of the
algebraic representation of Runge--Kutta methods, following Butcher
\cite{Butcher2008_book}.
This includes the concept of effective order and a list of effective
order conditions.
Section~\ref{sec:ExRK_barrier} proves an order barrier for effective
order methods with strictly positive weights, a consequence of which
is the non-existence of explicit SSP Runge--Kutta methods of effective
order five.
Section~\ref{sec:optimal_ESSPRK} presents effective order SSPRK
methods found by numerical search, some of which are established as
optimal.
Starting and stopping methods are also discussed.
The paper concludes with numerical experiments in
Section~\ref{sec:numerics} and conclusions in
Section~\ref{sec:Conclusion}.
