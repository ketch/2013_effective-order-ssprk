\section{Introduction}\label{sec:Intro}

\quad The solutions of nonlinear hyperbolic partial differential equations contain discontinuities propagating at finite speeds, even when the initial conditions are smooth. The challenge of the numerical solution of these systems is twofold. It is desirable that the approximation is of high accuracy in regions where the solution is smooth and that the discontinuities are captured without exhibiting any oscillations or overshoots.

In the last twenty years major developments have been made in this area by the introduction of strong stability preserving (SSP) time discretizations (formerly TVD discretizations \cite{Gottlieb1998}). The SSP methods solve ordinary differential equation systems resulting from a spatial discretization of the time-depending partial differential equations by maintaining certain strong stability properties. These stability properties of the original PDE problem are preserved by a suitable spatial discretization coupled with first order time integration. Thus, SSP methods are simply high order time discretiazations that guarantee the same stability preservation.

% In this paper we present the combination of effective order theory and the SSP theory for Runge-Kutta methods, underlining the impact of the effective order interpretation on SSP methods.

We examine the SSP properties of explicit Runge--Kutta methods of \emph{effective order}. These methods use special starting and stopping procedures which perturb the initial and final solution in such a way that a method can achieve an order of accuracy higher than its classical design order. This allows the construction of high order SSP Runge--Kutta schemes by using low-order SSP Runge--Kutta methods. We are able to find effective order four methods with four stages, which is not possible for classical order four explicit Runge--Kutta methods with four stages. However, the fifth-order barrier for explicit SSP Runge--Kutta can not be overcome. This is  establish later as a corollary of a more general theorem. Our results are also interesting in that most of the methods we find are optimal in the sense that they achieve the theoretical upper bounds on their SSP coefficients.

The rest of the paper is organized as follows. Section~\ref{sec:SSP} reviews Runge--Kutta methods and the concept of strong stability preserving methods.  Section~\ref{sec:Algebraic_RK} presents a brief overview of the algebraic representation of Runge--Kutta methods, following Butcher \cite{Butcher2008_book}. This includes the concept of effective order and a list of effective order conditions. Section~\ref{sec:ESSPRK_barrier} presents an order barrier for effective order methods with strictly positive weights. Section~\ref{sec:Optimal_ESSPRK} presents the effective order SSPRK methods found by numerical search and in some cases established as optimal. Starting and stopping methods are also discussed. \todo{starting/stopping is a new section or not?}\yianniscomment{I think is better starting/finishing methods to be in the same section as the main method. Some bits about optimization problem will be removed to section 2.} The paper concludes with numerical experiments in Section~\ref{sec:Numerics} and conclusions in Section~\ref{sec:Conclusion}. Proofs are included in Appendix~\ref{appendixA}.
