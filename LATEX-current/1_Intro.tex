\section{Introduction}\label{sec:Intro}

The solutions of nonlinear hyperbolic partial differential equations (PDEs) contain discontinuities propagating at finite speeds, even when the initial conditions are smooth.
The challenge of the numerical solution of these systems is twofold.
It is desirable that the approximation is of high accuracy in regions where the solution is smooth and that the discontinuities are captured without exhibiting any oscillations or overshoots.

Many numerical methods are based on a method-of-lines approach where the problem is first discretized with a spatial discretization to yield a system of ODEs.
The spatial discretization is often chosen to ensure certain strong stability properties of the original PDE problem (e.g., max-norm monotonicity, free of oscillations, positivity, etc {\bf [One cite each]}) are preserved \emph{when coupled with first-order Forward Euler time integration}.
Often the combination is subject to a time step size restriction.
Strong stability preserving (SSP) time discretizations (formerly TVD discretizations \cite{Gottlieb1998}) are high-order time discretiazations that guarantee the same stability preservation, with a possibly different step-size restriction.

% In this paper we present the combination of effective order theory and the SSP theory for Runge-Kutta methods, underlining the impact of the effective order interpretation on SSP methods.

We examine the SSP properties of explicit Runge--Kutta methods of \emph{effective order}.
These methods use special starting and stopping procedures which perturb the initial and final solution in such a way that a method can achieve an order of accuracy higher than its classical design order.
This allows the construction of high order SSP Runge--Kutta schemes by using low-order SSP Runge--Kutta methods.
We are able to find effective order four explicit SSP Runge--Kutta methods with four stages, which is not possible for classical order four (with four stages). However, the fifth-order barrier for explicit SSP Runge--Kutta \cite{Ruuth2002} cannot be overcome.
This is  established later as a corollary of a more general theorem.
Our results are also interesting in that most of the methods we find are optimal in the sense that they achieve the (previously) theoretical upper bounds on their SSP coefficients.

The rest of the paper is organized as follows. Section~\ref{sec:SSP} reviews Runge--Kutta methods and the concept of strong stability preserving methods.  Section~\ref{sec:Algebraic_RK} presents a brief overview of the algebraic representation of Runge--Kutta methods, following Butcher \cite{Butcher2008_book}. This includes the concept of effective order and a list of effective order conditions. Section~\ref{sec:ExRK_barrier} proves an order barrier for effective order methods with strictly positive weights, a consequence of which is the non-existence of explicit SSP Runge--Kutta methods of effective order five. Section~\ref{sec:Optimal_ESSPRK} presents the effective order SSPRK methods found by numerical search and in some cases established as optimal. Starting and stopping methods are also discussed.
The paper concludes with numerical experiments in Section~\ref{sec:numerics} and conclusions in Section~\ref{sec:Conclusion}.
