\section{Introduction}\label{sec:Intro}
Strong stability preserving time discretization methods were originally
developed for the solution of nonlinear hyperbolic 
partial differential equations (PDEs).  Solutions of such PDEs may 
contain discontinuities even when the initial conditions are smooth.
Many numerical methods for their solution are based on a method-of-lines approach 
in which the problem is first discretized in space to yield a system of ODEs. 
The spatial discretization is often chosen to ensure the solution is total variation diminishing (TVD),
in order to avoid the appearance of spurious oscillations near discontinuities,
\emph{when coupled with first-order forward Euler time integration}.
Strong stability preserving (SSP) time discretizations (also known as TVD
discretizations \cite{Gottlieb/Shu:1998}) are high-order time
discretizations that guarantee the TVD property (or other convex functional bounds), with a
possibly different step-size restriction \cite{Gottlieb2011a}.
Section~\ref{sec:SSP} reviews Runge--Kutta methods and the concept of
strong stability preserving methods.

Explicit SSP Runge--Kutta methods cannot have order greater
than four \cite{Ruuth2002}.  However, a Runge--Kutta method may 
achieve an effective order of accuracy higher than its classical order 
by the use of special starting and stopping procedures.
The conditions for a method to have effective order $q$ are 
in general less restrictive than the conditions for a method
to have classical order $q$.
Section~\ref{sec:Algebraic_RK} presents a brief overview of the
algebraic representation of Runge--Kutta methods, following Butcher
\cite{Butcher2008_book}.
This includes the concept of effective order and a list of effective
order conditions.

We examine the SSP properties of explicit Runge--Kutta methods whose
effective order is greater than their classical order.
Previous studies of SSP Runge--Kutta method have considered only the
classical order of the methods.
Three natural questions are:
\begin{itemize}
    \item Can an SSP Runge--Kutta method have effective order of accuracy greater than four?
    \item If we only require methods to have {\em effective} order $q$, is it possible to achieve larger
            SSP coefficient compared to methods with {\em classical} order $q$?
    \item SSP Runge--Kutta methods of order four require at least five stages.  Can SSP methods of 
          effective order four have fewer stages?
          
\end{itemize}
We show in Section \ref{sec:ExRK_barrier} that the answer to the first question is negative.
We answer the second question by numerically solving the problem of optimizing
the SSP coefficient over the class of methods with effective order $q$;
see Section~\ref{sec:optimal_ESSPRK}.  
Most of the methods we find are shown to be optimal, as they achieve a certain theoretical
upper bound on the SSP coefficient that is obtained by considering only
linear problems \cite{Kraaijevanger1986}.
We answer the last question affirmatively
by construction, also in Section~\ref{sec:optimal_ESSPRK}.

%Explicit SSP Runge--Kutta methods of order four require at 
%least five stages \cite{Gottlieb/Shu:1998}. 
%In contrast, we construct explicit SSP Runge--Kutta methods
%of effective order four with only four stages.
%Following this result, we had hoped to overcome the fifth-order
%barrier for explicit SSP Runge--Kutta methods \cite{Ruuth2002}; instead we
%prove that the barrier also holds for SSP methods of effective
%order five.
The paper concludes with numerical experiments in
Section~\ref{sec:numerics} and conclusions in
Section~\ref{sec:Conclusion}.
