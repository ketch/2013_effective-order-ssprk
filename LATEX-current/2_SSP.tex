\section{Strong stability preserving Runge--Kutta methods}\label{sec:SSP}
Strong stability preserving time-stepping methods were originally introduced
for time integration of hyperbolic conservation laws
\cite{Shu/Osher:1988} 
\begin{align}\label{eq:ode_system}
	\bm{U}_t + \nabla \cdot \bm{f}(\bm{U}) = 0,   
\end{align}
with appropriate initial and boundary conditions.
A spatial discretization gives the system of ODEs
\begin{align*}
    \bm{u}'(t) = \bm{F}(t,\bm{u}(t)),
\end{align*}
where $\bm{u}$ is a vector of continuous-in-time grid values approximating 
the solution $\bm{U}$ at discrete grid points and $\bm{F}$ is the spatial 
discretization. \yiannistodo{I think this was Colin's comment: TODO: do we need add something about F being more general than just HCL?}
A time discretization then produces a sequence of solutions
$\bm{u}^{n} \approx \bm{u}(t_n)$.

Assume that the solution of \eqref{eq:ode_system} satisfies a monotonicity 
property in some norm, seminorm, or convex functional.
Then many popular spatial discretizations are designed such that for a 
suitable class of problems the solution $\bm{u}^{n}$ computed with the
forward Euler scheme
\begin{align}\label{eq:forwardEuler}
    \bm{u}^{n+1} = \bm{u}^{n} + \Dt\bm{F}(t,\bm{u}^{n}),
\end{align}
is non-increasing in the same norm, seminorm, or convex functional:
\begin{align*}
    \|\bm{u}^{n+1}\| \le \|\bm{u}^n\|, \; \forall \bm{u},
\end{align*}
for some time-step restriction $\Dt \leq \Dt_{\text{FE}}$. \davidtodo{Does $\Dt_\textup{FE}$ depend on $u$?}
If this is the case, then an SSP method also generates solutions which are 
non-increasing in time, under a modified time-step restriction. \davidtodo{The term non-increasing could be misunderstood; maybe say non-increasing in time?}
\begin{definition}[Strong Stability Preserving]
	Suppose $\bm{F}$ is a spatial discretization which satisfies
	\begin{align*}
		\|\bm{u}^{n} + \Dt\bm{F}(\bm{u}^{n})\| \le \|\bm{u}^n\|, \forall \bm{u},
	\end{align*}
	for suitably-restricted time steps $0 \le \Dt \le \Dt_{\text{FE}}$,
	in a norm, seminorm, or convex functional $\|\cdot\|$ of interest.
	Then a time-stepping method is said to be \emph{strong stability
  	preserving} with \emph{SSP coefficient} $\sspcoef > 0$ if it
	generates a sequence of solution values $\bm{u}^n$ such that
	\begin{align*}
  		\|\bm{u}^{n+1}\| \le \|\bm{u}^n\|,
	\end{align*}
	for a time-step restriction
	\begin{align*}
		\Dt \leq \sspcoef \Dt_{\text{FE}}.
	\end{align*}
\end{definition} \davidtodo{In this definition, it seems that the SSP property of a method depends on the choice of $F$.}

\davidtodo{Should this really be a subsection?  We never refer to it.}
The SSP coefficient $\sspcoef$ is a property only of the particular 
time-stepping method and quantifies the allowable time step size relative 
to that of the forward Euler method.
Generally we want the SSP coefficient to be as large as possible for efficiency.
To allow a fair comparison of different explicit methods, we consider the 
\emph{effective SSP coefficient}
\begin{align*}
	\ceff = \frac{\sspcoef}{s}.
\end{align*}
Note that the use of the word \emph{effective} here is unrelated to the 
concept of \emph{effective order} introduced in Section~\ref{sec:Algebraic_RK}.

\subsection{Optimal SSP schemes}\label{subsec:Optimal_SSPRK}
An explicit $s$-stage Runge--Kutta method generates a sequence of approximate solutions $\bm{u}^n \approx \bm{u}(t_n)$ via an update formula
\begin{align*}
	\bm{u}^{n+1} &= \bm{u}^{n} + \Dt \sum_i^s b_i \bm{F}(t_n + c_i \Dt,\bm{Y}_i), \\
	\text{where} \quad \bm{Y}_i &= \bm{u}^{n} + \Dt \sum_j^{i-1} a_{ij} \bm{F}(t_n + c_j \Dt,\bm{Y}_j)
\end{align*}
and $c_i = \sum_j^sa_{ij}$. \davidtodo{This isn't very clear.}
The accuracy of the method is determined by comparing terms of Taylor 
expansions of the numerical and true solution. 
Also the stability function is a fraction of polynomials on the coefficients of
the method.
Thus, we can fully characterize a Runge--Kutta method by the triple 
$(A,\bm{b},\bm{c})$ \cite{Butcher2008_book}.
%The accuracy and stability of the method depend on the coefficients $(A,\bm{b},\bm{c})$
%\cite{Butcher2008_book}.

We say that an SSP Runge--Kutta method is optimal if it has the largest 
possible SSP coefficient for a given order and a given number of stages.
The search for these optimal methods was originally based on
expressing the Runge--Kutta method as combinations of forward Euler
steps in the Shu--Osher form and solving a nonlinear optimization
problem \cite{Gottlieb/Shu:1998, Gottlieb2001, Spiteri2003a, Spiteri2003b,
  Ruuth2004, Ruuth:2006}.
However, the SSP coefficient is related to the 
\emph{radius of absolute monotonicity} \cite{Kraaijevanger1991} and, 
for irreducible Runge--Kutta methods, the two are equivalent 
\cite{Ferracina2004, Higueras2004}.
This gives  simplified algebraic characterization of the SSP coefficient. 
\cite{Ferracina2005}
The optimization problem of finding optimal SSP Runge--Kutta methods
can be written in terms of the coefficients $A$ and $\bm{b}$ as
follows:
\begin{equation}\label{eq:SSP_opt}
    \max_{A, \bm{b}, r} \; r, \quad \text{subject to} \quad \left\{
                                                 \begin{array}{ll}
                                                   K(I + rA)^{-1} \geq 0 \\
                                                   \bm{e}_{s+1} - rK(I + rA)^{-1}\bm{e}_{s} \geq 0 \\
                                                   \Phi(K) = 0.
                                                 \end{array}
                                               \right.
\end{equation}
Here
\begin{equation*}
    K = \left(
            \begin{array}{c}
                     A              \\
                     \bm{b}^{\texttt{T}}
            \end{array}
         \right),
\end{equation*}
while $\bm{e}_s$ denotes the vector of ones of length $s$,
and \( \Phi(K) \) represents the  order conditions.
The inequalities are understood component-wise.

Useful upper bounds for the above optimization problem can be obtained 
by considering an important relaxation. 
In the relaxed problem, the method is required to be accurate and strong 
stability preserving only for linear, constant-coefficient initial value problems. 
This leads to a reduced set of order conditions and a relaxed absolute 
monotonicity condition.
We denote the SSP coefficient of a method for linear problems by $\clin$; 
clearly for any method
\begin{align*}
	\sspcoef\le\clin,
\end{align*}
and the same inequality holds for the optimal coefficients over a given class.
Exact optimal values of $\clin$ are known for many classes of methods; for
example see \cite{Kraaijevanger1986,ketcheson2009a}.

Following \cite{Ketcheson2008, Ketcheson/Macdonald/Gottlieb:2009}, 
we will use the optimization problem \eqref{eq:SSP_opt} to perform a 
numerical search for optimal explicit SSP Runge--Kutta methods. 
However, we first need to define the order conditions $\Phi(K)$ for methods 
of effective order.
This is discussed in the next section.
