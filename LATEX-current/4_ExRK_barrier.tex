\section{A barrier on the effective order of explicit Runge--Kutta methods with positive weights}\label{sec:ExRK_barrier}
In the next section, we numerically optimize the SSP coefficient for effective order Runge-Kutta
methods.  The results in this section demonstrate that we need only consider methods with effective
order up to four.  We state the results and then give their proofs.

\davidcomment{I removed the bit here because it didn't seem any simpler than
just reading the proof.}
%As Table~\ref{tab:Effective_oc} shows, if we consider an $s$-stage
%RK($s$,$5$,$2$) method, then the coefficients of the main method
%depend on $\beta_2$.
%In particular, the main method must satisfy
%\begin{equation}\label{eq:2nd_order_con}
%    \frac{1}{2}\alpha_3 - \frac{1}{6} = \beta_2 \; \text{ and } \; \frac{1}{4}\alpha_9 - \alpha_{10} + \alpha_{13} = (\beta_2)^2.
%\end{equation}
%For classical order $p \geq 3$ an $s$-stage RK($s$,$5$,$p$) method must satisfy
%\begin{equation}\label{eq:higher_order_con}
%    \frac{1}{4}\alpha_9- \alpha_{10} + \alpha_{13} = 0,
%\end{equation}
%since $\beta_2 = 0$.
%If we require positive weights then the equations \eqref{eq:2nd_order_con},
%$\alpha_4 = \frac{1}{6}$ and $\alpha_1 = 1$ can not be solved simultaneously.
%Also the equation \eqref{eq:higher_order_con} is not compatible with the main method having stage order one.
%This immediately suggests the non existence of fifth-effective order explicit Runge-Kutta methods with positive weights.
%These results are established as follows.

\begin{theorem}\label{thm:positiveb}
    Let $M$ denote a Runge--Kutta method with positive weights: $\bm{b}>0$.
    Then $M$ has effective order of at most four.
\end{theorem}
The proof of Theorem~\ref{thm:positiveb} is given below.

\begin{lemma}\label{lem:positiveb}(see \cite[Theorem~4.2]{Kraaijevanger1991},\cite[Lemma 4.2]{Ruuth2002})
Let $M$ denote an irreducible Runge--Kutta method with $\sspcoef>0$.  Then $M$ has positive weights:
$\bm{b}>0$.
\end{lemma}

\begin{corollary}\label{cor:no-ssp-5}
    Let $M$ denote a Runge--Kutta method with $\sspcoef>0$.
    Then $M$ has effective order of at most four.
\end{corollary}
This follows immediately from Theorem~\ref{thm:positiveb} and Lemma~\ref{lem:positiveb}.

\begin{remark}
  From Theorem~\ref{thm:positiveb} and \cite[Theorem~4.1]{dahlquist2006}, we
  may also conclude that any Runge-Kutta method with positive radius of circle contractivity
  has effective order at most four.
\end{remark}

The proof of Theorem \ref{thm:positiveb} makes use of the following lemma.
\begin{lemma}\label{Davids_lemma}
	Let $\bm{b},\bm{v} \in \mathbb{R}^{n}$ be given such that
    \begin{subequations}\label{eq:DavidsLemma}
    		\begin{align}
    			b_i & > 0 \mbox{ for all } i, \label{eq:DavidsLemma_a} \\
    			\sum_{i=1}^n b_i & = 1, \label{eq:DavidsLemma_b} \\
    			\sum_{i=1}^n b_i v_i^2 & = \left(\sum_{i=1}^n b_i v_i \right)^2. \label{eq:DavidsLemma_c}
    		\end{align}
    	\end{subequations}
    	Then all $v_i$ are equal but at most one; in other words, there exists $\mu \in \mathbb{R}$ and an integer $k$ such that $v_i = \mu$ for all $i \ne k$.
\end{lemma}

\begin{proof}
    Let \( k \) be an arbitrary integer between \( 1 \) and \( n \). Then by collecting terms in powers of \( v_k \),
%\eqref{eq4.3c} can be written
\eqref{eq:DavidsLemma_c} can be written
\begin{equation*}
    %b_k v_k^2 + \sum_{i\ne k} b_i v_i^2 - b_k^2 v_k^2 - 2 b_k v_k \sum_{i\ne k} b_i v_i
    %        - \left(\sum_{i\ne k} b_i v_i \right)^2 = 0 \\
  b_{k}(1-b_{k})v_{k}^{2} - 2b_{k}v_{k}\sum_{i \neq k}b_{i}v_{i} + \sum_{i \neq k}b_{i}v_{i}^{2} - \left(\sum_{i \neq k}b_{i}v_{i}\right)^{2} = 0.
\end{equation*}
This is a quadratic equation in \( v_{k} \) whose roots are real if and only if
\begin{equation*}
    4b_{k}^{2}\left(\sum_{i \neq k}b_{i}v_{i}\right)^{2} - 4b_{k}(1-b_{k})\left(\sum_{i \neq k}b_{i}v_{i}^{2} - \left(\sum_{i \neq k}b_{i}v_{i}\right)^{2}\right) \geq 0.
\end{equation*}
Expanding and canceling terms yields
\begin{equation*}
    (1-b_{k})\sum_{i \neq k}b_{i}v_{i}^{2} - \left(\sum_{i \neq k}b_{i}v_{i} \right)^{2} \leq 0.
\end{equation*}
By \eqref{eq:DavidsLemma_b}, \( 1-b_{k} = \sum_{j \ne k}b_{j} \), so we have
\begin{equation*}
    \sum_{j \neq k}b_{j}\sum_{i \neq k}b_{i}v_{i}^{2} - \sum_{j \neq k}b_{j}v_{j}\sum_{i \neq k}b_{i}v_{i} \le 0.
\end{equation*}
Noting that the terms corresponding to \( i = j \) in the two double sums cancel gives
%\begin{equation*}
%    \sum_{j \neq k}b_{j}\sum_{i \neq k,j}b_{i}v_{i}^{2} - \sum_{j \neq k}b_{j}v_{j}\sum_{i \neq j,k}b_{i}v_{i} \leq 0,
%\end{equation*}
%or
\begin{equation*}
    \sum_{j \neq k}b_{j}\sum_{i \neq k,j}b_{i}v_{i}(v_{i} - v_{j}) \leq 0.
\end{equation*}
Adding the left hand side to itself, but with \( i,j \) reversed, yields
\begin{equation*}
    \sum_{j \neq k}b_{j}\sum_{i \neq k,j}b_{i}v_{i}(v_{i} - v_{j}) - \sum_{i \neq k}b_{i}\sum_{j \neq k,i}b_{j}v_{j}(v_{i} - v_{j}) \leq 0.
\end{equation*}
This simplifies to
\begin{equation*}
    \sum_{j \neq k}\sum_{i \neq k,j}b_{j}b_{i}(v_{i} - v_{j})^{2} \leq 0.
\end{equation*}
Together with \eqref{eq:DavidsLemma_a}, this implies that \( v_{i} = v_{j} \) for all \( i,j \neq k \).
\end{proof}

\begin{proof}[Proof of Theorem~\ref{thm:positiveb}]
    {\bf This proof needs some attention.}

    Any method of effective order five must have classical order at least two {\bf Cite?}.
    Thus it is sufficient to show that any method with all non-negative weights
    cannot satisfy the conditions of effective order five and classical order two.

    Assume $\bm{b}>0$ and consider the order conditions for effective order five and classical order two for the main method, listed in Table~\ref{tab:effective_OCs}.  We make use of the following subset of those conditions:
    \begin{subequations}\label{eq4.4}
        \begin{align}
            \bm{b}^{T}\bm{e} &= 1 \label{eq4.4a} \\
            \bm{b}^{T}A\bm{c} &= \frac{1}{6} \label{eq4.4b} \\
            \frac{1}{2}\bm{b}^{T}\bm{c}^{2} - \frac{1}{6} &= \beta_{2} \label{eq4.4c} \\
            \frac{1}{4}\bm{b}^{T}\bm{c}^{4} - \bm{b}^{T}C^{2}A\bm{c} + \bm{b}^{T}(A\bm{c})^{2} &= \beta_{2}^{2}, \label{eq4.4d}
        \end{align}
    \end{subequations}
    where the powers on vectors are understood component-wise. 
    We also make use of the following two quantities:
    \begin{align*} 
        \bm{v} = \frac{1}{2}\bm{c}^{2} - A\bm{c} \\
        \bm{w} = \bm{v}^{2} - \beta_{2}\bm{v}.
    \end{align*}
    Note that, for any explicit method, $\bm{v}\ne 0$, since explicit methods
    cannot have stage order two \cite{Kraaijevanger1991}.

    First we show that $\beta_2 \ne 0$.  Suppose $\beta_2=0$.
    Then \eqref{eq4.4d} becomes
    \begin{equation*}
        \frac{1}{4}\bm{b}^{T}\bm{c}^{4} - \bm{b}^{T}C^{2}A\bm{c} + \bm{b}^{T}(A\bm{c})^{2} = 0,
    \end{equation*}
    or equivalently
    \begin{equation*}
        \bm{b}^{T}\bm{v}^{2} = 0,
    \end{equation*}
    which is a contradiction since $\bm{v}\ne0$ and $\bm{b}>0$.

    Substituting \eqref{eq4.4b} in \eqref{eq4.4c} and collecting terms gives 
    \begin{align} \label{eq:btv}
        \beta_{2} = \bm{b}^{T}\bm{v}.
    \end{align}
    Squaring yields
    \begin{equation*}%\label{eq4.5}
        \beta_{2}^{2} = \bm{b}^{T}\Bigl(\frac{1}{2}\bm{b}^{T}\bm{c}^{2} - \frac{1}{6}\Bigr)\Bigl(\frac{1}{2}\bm{c}^{2} - A\bm{c}\Bigr),
    \end{equation*}
    %Rewriting \eqref{eq4.4d} and subtracting from \eqref{eq4.5} gives
    and subtracting \eqref{eq4.4d} gives
    \begin{equation*}
        \bm{b}^{T}\biggl(\Bigl(\frac{1}{2}\bm{c}^{2} - A\bm{c}\Bigr)^{2} - \Bigl(\frac{1}{2}\bm{b}^{T}\bm{c}^{2} - \frac{1}{6}\Bigr)\Bigl(\frac{1}{2}\bm{c}^{2} - A\bm{c}\Bigr)\biggr) = 0.
    \end{equation*}
    This can be written as
    \begin{equation}\label{eq4.6}
        \bm{b}^{T}\bm{w} = 0.
    \end{equation}
    Since $\bm{b}>0$, then \eqref{eq4.6} implies that either \( \bm{w} = \bm{0} \),
    or there is a least one negative and at least one positive element in \( \bm{w} \).

\paragraph{Case 1: \( \bm{w} = \bm{0} \).}

We have \( v_{i}^{2} - \beta_{2}v_{i} = 0 \) for all \( i \in \{1,\dots,s\} \), so either \( v_{i} = 0 \) or \( v_{i} = \beta_{2} \).
%The case that \( \bm{v} = 0 \) is equivalent with a Runge-Kutta method to have stage order two. This means that for a particular time \( t = t^{n} + c_{i}\Delta t \), for \( i \in \{1, \dots, s\} \), the stage \( \textbf{Y}_{i} \) will be at least a second order approximation to the solution at that time.
%For an explicit Runge-Kutta method, the stage \( \textbf{Y}_{2} = \bm{u}^{n} + \Dt\bm{F}(\bm{u}^{n}) \) cannot have stage order more than one, hence not all the components of \( \bm{v} \) are zero. 
Let \( I = \{ i \in\{1,\dots,s\} | v_{i} = \beta_{2} \} \); then \( v_{j} = 0 \) for \( j \neq I \). 
Then \eqref{eq:btv} implies 
$$ \beta_{2} = \sum_{i=1}^{s}b_{i}v_{i} = \sum_{i \in I}b_{i}\beta_{2} = \beta_{2}\sum_{i \in I}b_{i}.$$
Note that \( v_{1} = 0 \) because the first row of matrix \( A \) is identically zero. 
Thus $\sum_{i\in I}b_i < \sum_{i=1}^s b_i$, so \eqref{eq4.4a} implies
$$\beta_{2}\sum_{i \in I}b_{i}< \beta_{2},$$
which is a contradiction.

\paragraph{Case 2: \( \bm{w} \neq \bm{0} \).}
Then there exist $i,j$ such that $w_i<0<w_j$.
     We will show that \( 0 < v_{i} < \beta_{2} < v_{j} \).

    Observe that $w_i<0$ means \( v_{i}^{2} < \beta_{2}v_{i} \). Therefore,
    \begin{equation}\label{eq4.7}
        (\beta_{2})^{2} = \sum_{j=1}^{s}b_{j}v_{j}^{2} < \sum_{j \neq i}^{s}b_{j}v_{j}^{2} + b_{i}\beta_{2}v_{i}.
    \end{equation}
    The inequality \eqref{eq4.7} can be written as
    \begin{equation}\label{eq4.8}
        \begin{split}
            \beta_{2}\sum_{j=1}^{s}b_{j}v_{j} &< \sum_{j \neq i}^{s}b_{j}v_{j}^{2} + \beta_{2}b_{i}v_{i} \\
            \beta_{2}\sum_{j \neq i}^{s}b_{j}v_{j} &< \sum_{j \neq i}^{s}b_{j}v_{j}^{2} \\
            \beta_{2}\sum_{j \neq i}^{s}b_{j}v_{j} &< (\beta_{2})^{2}.
        \end{split}
    \end{equation}
    Suppose that \( v_{i} < 0 \). Then
    \begin{equation*}
        \beta_{2} = \sum_{j=1}^{s}b_{j}v_{j} < \sum_{j \neq i}^{s}b_{j}v_{j}
    \end{equation*}
    From \eqref{eq4.8} we have
    \begin{equation*}
            \beta_{2}\beta_{2} < \beta_{2}\sum_{j \neq i}^{s}b_{j}v_{j} < (\beta_{2})^{2},
    \end{equation*}
    which gives a contradiction. Since \( v_{i} \neq 0 \), then \( v_{i} > 0 \) and in this case \( 0 < v_{i} < \beta_{2} \).

    Note that \( w_{j} > 0 \), means \( v_{j}^{2} > \beta_{2}v_{j} \). If \( v_{j} < 0 \), then \( v_{j} < \beta_{2} \) and \( v_{j} < 0 < v_{i} < \beta_{2} \). Hence \( \beta_{2} \) is larger than all components of \( \bm{v} \). This leads to a contradiction since
    \begin{equation*}
            \beta_{2} = \sum_{k=1}^{s}b_{k}v_{k} < \sum_{k=1}^{s}b_{k}\beta_{2} < \beta_{2}.
    \end{equation*}
    Since \( v_{j}\ne0 \), therefore \( v_{j} > 0 \) which gives \( v_{j} > \beta_{2} \).

    In summary, we have shown that
    \begin{equation}\label{eq4.9}
        v_{1} = 0 \quad \text{ and } \quad 0 < v_{i} < \beta_{2} < v_{j}, \text{ for } i,j \in \{2,\dots,s\}.
    \end{equation}
    By Lemma~\ref{Davids_lemma}, all \( v_{i} \) must be equal except for one.
    Let that component be \( v_{k} \). If \( k = 1 \), then \( v_{i} = \delta
    \) {\bf what is $\delta$?} for \( i,j \in \{2,\dots,s\} \). If \( k \neq 1
    \), then \( v_{i} = 0 \) for \( i \in \{2,\dots,s\}\setminus\{k\} \). Both
    cases contradict with \eqref{eq4.9}. %Thus, the equations \eqref{eq4.4}
    %cannot be solved simultaneously and no Runge-Kutta method of
    %effective order five and classical order two exists with positive weights.
    %{\bf what about classical order less than 2?}

\end{proof}
