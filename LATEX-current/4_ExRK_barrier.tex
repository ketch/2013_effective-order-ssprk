\section{A barrier on the effective order of explicit Runge-Kutta methods with positive weights}\label{sec:ExRK_barrier}

\indent As Table~\ref{tab:Effective_oc} shows, if we consider an $s$-stage RK($s$,$5$,$2$) method, then the coefficients of the main method depend on $\beta_2$. In particular, the main method must satisfy
\begin{equation}\label{eq:2nd_order_con}
    \frac{1}{2}\alpha_3 - \frac{1}{6} = \beta_2 \; \text{ and } \; \frac{1}{4}\alpha_9 - \alpha_{10} + \alpha_{13} = (\beta_2)^2.
\end{equation}
For classical order $p \geq 3$ an $s$-stage RK($s$,$5$,$p$) method must satisfy
\begin{equation}\label{eq:higher_order_con}
    \frac{1}{4}\alpha_9- \alpha_{10} + \alpha_{13} = 0,
\end{equation}
since $\beta_2 = 0$. As we will prove later in Theorem~\ref{thm:positiveb}, if we require positive weights the equations \eqref{eq:2nd_order_con}, $\alpha_4 = \frac{1}{6}$ and $\alpha_1 = 1$ can not be solved simultaneously. Also the equation \eqref{eq:higher_order_con} is not compatible with the main method having stage order one. This immediately suggests the non existence of fifth-effective order explicit Runge-Kutta methods with positive weights.

\begin{theorem}\label{thm:positiveb}
    Let $M$ denote a Runge--Kutta method with non-negative weights: $b\ge 0$.
    Then $M$ has effective order of at most four.
\end{theorem}
The proof of Theorem~\ref{thm:positiveb} is deferred to the appendix.

\begin{lemma}\label{lem:positiveb}(see \cite[Theorem~4.2]{Kraaijevanger1991},\cite[Lemma 4.2]{Ruuth2002})
Let $M$ denote an irreducible Runge--Kutta method with $\sspcoef>0$.  Then $M$ has positive weights:
$b>0$.
\end{lemma}

\begin{corollary}\label{cor:no-ssp-5}
    Let $M$ denote a Runge--Kutta method with $\sspcoef>0$.
    Then $M$ has effective order of at most four.
\end{corollary}
This follows immediately from Theorem~\ref{thm:positiveb} and Lemma~\ref{lem:positiveb}.

\begin{remark}
  From Theorem~\ref{thm:positiveb} and \cite[Theorem~4.1]{dahlquist2006}, we
  may also conclude that any Runge-Kutta method with positive radius of circle contractivity
  has effective order at most four.
\end{remark}


