\section{Explicit SSP Runge--Kutta methods have effective order at most four}\label{sec:ExRK_barrier}
The classical order of any explicit SSP Runge--Kutta method cannot be greater
than four \cite{Ruuth2002}.
It turns out that the effective order of any explicit SSP Runge--Kutta method 
also cannot be greater than four, although the proof of this result is more
involved.
%As Table~\ref{tab:Effective_oc} shows, if we consider an $s$-stage
%RK($s$,$5$,$2$) method, then the coefficients of the main method
%depend on $\beta_2$.
%In particular, the main method must satisfy
%\begin{equation}\label{eq:2nd_order_con}
%    \frac{1}{2}\alpha_3 - \frac{1}{6} = \beta_2 \; \text{ and } \; \frac{1}{4}\alpha_9 - \alpha_{10} + \alpha_{13} = (\beta_2)^2.
%\end{equation}
%For classical order $p \geq 3$ an $s$-stage RK($s$,$5$,$p$) method must satisfy
%\begin{equation}\label{eq:higher_order_con}
%    \frac{1}{4}\alpha_9- \alpha_{10} + \alpha_{13} = 0,
%\end{equation}
%since $\beta_2 = 0$.
%If we require positive weights then the equations \eqref{eq:2nd_order_con},
%$\alpha_4 = \frac{1}{6}$ and $\alpha_1 = 1$ can not be solved simultaneously.
%Also the equation \eqref{eq:higher_order_con} is not compatible with the main method having stage order one.
%This immediately suggests the non existence of fifth-effective order explicit Runge-Kutta methods with positive weights.
%These results are established as follows.
%\begin{theorem}\label{cor:no_SSP_5}
%    Let $M$ denote an explicit Runge--Kutta method with $\sspcoef>0$.
%    Then $M$ has effective order at most four.
%\end{theorem}
%
%In this section we prove an even stronger result,
%which is stated in the following lemma.
%\begin{lemma}\label{thm:effective_barrier}
%	Any explicit Runge--Kutta method with positive weights $\bm{b} > \bm{0}$
%        has effective order at most four.
%\end{lemma}
%
%Theorem \ref{cor:no_SSP_5} follows immediately from Lemma \ref{thm:effective_barrier}
%and the following well-known result.
%\begin{lemma}\label{lem:positive_b}(see \cite[Theorem~4.2]{Kraaijevanger1991},\cite[Lemma 4.2]{Ruuth2002})
%	Any irreducible Runge--Kutta method with positive SSP coefficient $\sspcoef>0$
%	must have positive weights $\bm{b} > \bm{0}$.
%\end{lemma}
%
%\begin{remark}
%	Using Lemma~\ref{thm:effective_barrier} and \cite[Theorem~4.1]{dahlquist2006}, 
%  	we may also conclude that any explicit Runge--Kutta method with positive radius of
%  	circle contractivity has effective order at most four.
%\end{remark}
We begin by recalling a well-known result.
\begin{lemma}\label{lem:positive_b}(see \cite[Theorem~4.2]{Kraaijevanger1991},\cite[Lemma 4.2]{Ruuth2002})
	Any irreducible Runge--Kutta method with positive SSP coefficient $\sspcoef>0$
	must have positive weights $\bm{b} > \bm{0}$.
\end{lemma}
Irreducibility \cite{dahlquist2006} is technically important in this
result and those that follow because a reducible SSP method might not
have positive weights (but it would be reducible to one that does, as
per the lemma).
The main result of this section is:
%to prove the non-existence of explicit SSP Runge-Kutta of effective order more than four,
%as stated in Theorem~\ref{the:no_SSP_5}. 
\begin{theorem}\label{thm:effective_barrier}
	Any explicit Runge--Kutta method with positive weights $\bm{b} > \bm{0}$ 
	has effective order at most four.
\end{theorem}

%The proof of Theorem~\ref{thm:effective_barrier} is deferred to
%the end of this section.

%We can now prove Theorem~\ref{thm:effective_barrier}:
\begin{proof}%[Proof of Theorem~\ref{thm:effective_barrier}]
	Any method of effective order five must have classical order at least two
	(see \cite{Butcher2008_book} or Table~\ref{tab:effective_OCs}).
    Thus it is sufficient to show that any method with all positive weights
    cannot satisfy the conditions of effective order five and classical order two.

    Let $(A,\bm{b},\bm{c})$ denote the coefficients of an explicit Runge--Kutta method with
    effective order at least five, classical order at least two, and positive 
    weights $\bm{b} > \bm{0}$.
    The effective order five and classical order two conditions
    (see Table~\ref{tab:effective_OCs} with $q=5$ and $p=2$) include the following:
    \begin{subequations}\label{eq:theorem_cond}
    		\begin{align}
    			\bm{b}^T\bm{e} & = 1, \label{eq:theorem_cond_a} \\
             	\bm{b}^TA\bm{c} &= \frac{1}{6}, \label{eq:theorem_cond_b} \\
            	\frac{1}{2}\bm{b}^T\bm{c}^2 - \frac{1}{6} &= \beta_2, \label{eq:theorem_cond_c} \\
            	\frac{1}{4}\bm{b}^T\bm{c}^4 - \bm{b}^TC^2A\bm{c} + \bm{b}^T(A\bm{c})^2 &= \beta_2^2, \label{eq:theorem_cond_d}
        	\end{align}
	\end{subequations}
	where the powers on vectors are understood component-wise. 
	Let
	%\begin{align*}
		$\bm{v} = \frac{1}{2}\bm{c}^2 - A\bm{c}$.
	%\end{align*}
	Then substituting \eqref{eq:theorem_cond_b} into \eqref{eq:theorem_cond_c} and expressing \eqref{eq:theorem_cond_d} in terms of $\bm{v}$ gives
        %\begin{subequations}
	\begin{align*}
		\bm{b}^T\bm{v} &= \beta_2, \\% \label{eq:btv} \\
        %\end{equation}
	%Also, \eqref{eq:theorem_cond_d} can be expressed as
	%\begin{equation}
		\bm{b}^T\bm{v}^2 &= \beta_2^2. % \label{eq:btv2}
	\end{align*}
        %\end{subequations}
  Each of %\eqref{eq:btv} and \eqref{eq:btv2}
  these is a strictly convex combination.
  % (by \eqref{eq:theorem_cond_a} and $\bm{b} > 0$).
  Jensen's inequality (with a strictly convex function, as is the case with the
  square function here) then states $\bm{b}^T \bm{v}^2 \leq (\bm{b}^T
  \bm{v})^2$ with equality if and only if all
  all components of $\bm{v}$ are equal
  %$v_i = \mu, i=1,\ldots,n$
  \cite[Theorem 12, pg 31]{Bullen:inequalities}.
  %Thus all components of $\bm{v}$ are equal.
  Now $v_1 = 0$ for every explicit method so we deduce that $\bm{v}=0$.
  That implies the method has stage order two, which is not possible for
  explicit methods \cite{Ruuth2002}.
  % \cite{Hairer1991_book}.
  % Ruuth2002 is cited here but this is a well-known result.
  This contradiction completes the proof.
\end{proof}

\begin{corollary}\label{cor:no_SSP_5}
    Let $M$ denote an irreducible explicit Runge--Kutta method with $\sspcoef>0$.
    Then $M$ has effective order at most four.
\end{corollary}
\begin{proof}
	The proof follows immediately from Lemma~\ref{lem:positive_b} and 
	Theorem~\ref{thm:effective_barrier}.
\end{proof}

\begin{remark}
    It is worth noting here an additional result that 
    follows directly from what we have proved.
    Using Theorem~\ref{thm:effective_barrier} and \cite[Theorem~4.1]{dahlquist2006}, 
    it follows that any irreducible explicit Runge--Kutta method with positive radius of
    circle contractivity has effective order at most four.
\end{remark}


