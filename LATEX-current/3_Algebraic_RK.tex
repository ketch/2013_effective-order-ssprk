\section{The effective order of Runge--Kutta methods}\label{sec:Algebraic_RK}

The definition, construction, and application of methods with an
effective order of accuracy relies on the use of starting and stopping
methods.
%The successive application of these methods results in a scheme that attains
%higher order of accuracy than the order of its consisting methods.
Specifically, we consider a \emph{starting method} $S$, a \emph{main
  method} $M$, and a \emph{stopping method} $S^{-1}$.
%: i.e., the inverse of $S$ which
%annihilates the work of the starting method (up to order $q$).
The successive use of these three methods results in a method $P =
S^{-1}MS$, which denotes the application method $S$, followed by
method $M$, followed by method $S^{-1}$.
We want $P$ to have order $q$, whereas $M$ might have lower classical
order $p < q$.
We then say $M$ has \emph{effective order} $q$.

When the method $P$ is used for $n$ steps,
$P^n = (S^{-1}MS)^n = (S^{-1}MS) \cdots (S^{-1}MS) (S^{-1}MS),$
it turns out that only $M$ need be used repeatedly, as in
$S^{-1} M^n S$,
because %, as suggested by the $S^{-1}$ notation,
$S S^{-1}$ leaves the solution unchanged up to order $q$.
The starting method introduces a perturbation to the solution,
followed by $n$ time steps of the main method $M$, and finally the
stopping method is used to correct the solution.
In Section~\ref{subsec:starting_stopping}, we propose alternative
starting and stopping procedures which allow the overall procedure to
be SSP.

The effective order of a Runge--Kutta method is defined in an abstract 
algebraic context introduced by Butcher \cite{Butcher1969} and developed 
further in \cite{Butcher1972, Hairer1974, Butcher1996, Butcher1998} and 
others.
We follow the book \cite{Butcher2008_book} in our description and
derivation of the effective order conditions.


\subsection{The algebraic representation of Runge--Kutta methods}\label{subsec:Algebraic_representation}

In Butcher's algebraic theory of Runge--Kutta methods
\cite{Butcher2008_book}, methods are viewed as elements in an group
$G$, consisting of real-valued functions on the set of rooted trees.
\textbf{\red todo: next sentence needs work?}
The connection between Runge--Kutta methods and functions in $G$ is
made via \emph{elementary weights} $\Phi(t)$, which are polynomial
expressions in the coefficients of a method, with each expression
related to a tree.
%Each elementary weight is an expression $\Phi(t)$ on a rooted tree
%$t$ \cite{Butcher2008_book}.
Table~\ref{tab:elementary_weights} lists these expressions for trees of
up to degree five; a general recursive formula can be found in
\cite[Definition 312A]{Butcher2008_book}.
For a function $\alpha \in G$ we write the values of the
elementary weights as $\alpha_{i} = \alpha(t_{i})$ for tree $t_{i}$.
%By convention $\alpha_0 = \alpha(t_{0}) = 1$, where $t_{0}$ denotes the empty tree.
%For example $\alpha_1 = \bm{b}^T\bm{e}$, $\alpha_2 = \bm{b}^T\bm{c}$ and so on.
A special element of the group $E \in G$ corresponds to the
(hypothetical) method which takes one exact step of the
solution.
The values of $E(t)$ are denoted $1/\gamma(t)$ and these are included in
Table~\ref{tab:elementary_weights}; (classical) order conditions
follow from comparing the elementary weights of a method with these
values.


Let $\alpha, \beta \in G$ correspond to Runge--Kutta methods $M_1$ and $M_2$
respectively.
%A multiplicative group operation $\alpha\beta$ can be defined
The application of method $M_1$ followed by method $M_2$ corresponds to
the multiplicative group operation $\alpha\beta$.\footnote{We write
	$M_2M_1$ to mean the application of $M_1$
	followed by the application of $M_2$
	(following matrix and operator ordering convention)
	but when referring to products of elements of $G$
        we use the reverse ordering ($\alpha\beta$)
	to match the convention in \cite{Butcher2008_book}.}
This is defined by partitioning the input tree and computing
over the resulting forest \cite[\S~383]{Butcher2008_book}.
\textbf{\red note removed defn, ok?}
%\begin{equation}\label{eq:Group_operation}
%       (\alpha\beta)(t) = \sum_{w \lhd t} \biggl(\prod_{v \in t \setminus w} \alpha(v)\beta(w)\biggr),
%\end{equation}
%where $w \lhd t$ indicates a subtree of $t$ which includes the
%root of $t$ and $w \setminus t$ indicates the forest induced
%by removing $w$ from $t$ \cite{Butcher2008_book}.
%Multiplicity in choosing $w$ must also be accounted for.



\begin{table}
	\centering
	\begin{smalltrees}
		\begin{tabular}{ccccccc}
    		\cline{1-3}\cline{5-7}
    		$i$ & tree $t_i$ & elementary weight & & $i$ & tree $t_i$ & elementary weight \\
    		\cline{1-3}\cline{5-7} \\[-10pt]
    		0 & $\emptyset$ \hspace{15pt}  & 1 & & 9 & \hspace{15pt} \tree{9} & $\bm{b}^T\bm{c}^4$\\
    		1 & \hspace{15pt}  \tree{1} &$\bm{b}^T\bm{e}$ & & 10 & \tree{10} \hspace{15pt} & $\bm{b}^TC^2A\bm{c}$ \\
    		2 & \tree{2} \hspace{15pt}  &$\bm{b}^T\bm{c}$ & & 11 & \hspace{15pt} \tree{11} & $\bm{b}^TCA\bm{c}^2$ \\
    		3 & \hspace{15pt}  \tree{3} & $\bm{b}^T\bm{c}^2$ & & 12 & \tree{12} \hspace{15pt} & $\bm{b}^TCA^2\bm{c}$ \\
    		4 & \tree{4} \hspace{15pt}  & $\bm{b}^TA\bm{c}$ & & 13 & \hspace{15pt} \tree{13} & $\bm{b}^T(A\bm{c})^2$ \\
    		5 & \hspace{15pt}  \tree{5} & $\bm{b}^T\bm{c}^3$ & & 14 & \tree{14} \hspace{15pt} & $\bm{b}^TA\bm{c}^3$ \\
    		6 & \tree{6} \hspace{15pt}  & $\bm{b}^TCA\bm{c}$ & & 15 & \hspace{15pt} \tree{15} & $\bm{b}^TACA\bm{c}$ \\
    		7 & \hspace{15pt}  \tree{7} & $\bm{b}^TA\bm{c}^2$ & & 16 & \tree{16} \hspace{15pt} & $\bm{b}^TA^2\bm{c}^2$ \\
    		8 & \tree{8} \hspace{15pt}  & $\bm{b}^TA^2\bm{c}$ & &  17 & \hspace{15pt} \tree{17} & $\bm{b}^TA^3\bm{c}$ \\
  		\end{tabular}
  \end{smalltrees}
  \caption{Elementary weights of trees up to order five for a 
  		Runge--Kutta method with Butcher tableau $(A,\bm{b},\bm{c})$. 
  		Here $C$ is a diagonal matrix with components 
  		$c_{i} = \sum_{j=1}^{i-1} a_{ij}$ and exponents of vectors 
  		represent component exponentiation.
  		\textbf{TODO: add $\gamma(t)$ values.}}
  \label{tab:elementary_weights}
\end{table}


Two Runge--Kutta methods $M_1$ and $M_2$, are equivalent up to order
$p$ if their corresponding elements in $G$, $\alpha$ and $\beta$, satisfy
$\alpha(t) = \beta(t)$, for every tree $t$ with $r(t) \leq p$,
where $r(t)$ denotes the order of the tree (number of vertices).
We denote this equivalence relation by
$$M_1 \equival{p} M_2,$$
(although this notation is not used in \cite{Butcher2008_book}.)
In this sense, methods have inverses: the product of $\alpha^{-1}$ and
$\alpha$ must match the identity method up to order $p$.
%Note that inverse methods up to order $p$ are not unique and inverse 
%methods of explicit methods need not be implicit.
We can then define the effective order of accuracy of a method $M$
with starting method $S$ and stopping method $S^{-1}$. % as $S^{-1}MS \equival{q} E$.
\begin{definition}\cite[\S~389]{Butcher2008_book}\label{def:Effective_order}
  Suppose $M$ is a Runge--Kutta method with corresponding $\alpha \in G$.
  Then the method $M$ is of effective order $q$ if there exists a method
  $S$ (with corresponding $\beta \in G$) such that
	\begin{equation}\label{eq:Effective_order_1}
		(\beta\alpha\beta^{-1})(t) = E(t), \; \text{for every tree with $r(t) \leq q$,}
	\end{equation}
        where $\beta^{-1}$ is an inverse of $\beta$ up to order $q$.
        Recall that $E$ represents one exact step of the solution.
\end{definition}

\subsection{Effective order conditions}\label{sec:effOrderCond}
For the main method $M$ to have effective order $q$ its coefficients,
and those of the starting and stopping methods, must satisfy a set of
algebraic conditions.
These \emph{effective order conditions} can be found
by rewriting \eqref{eq:Effective_order_1} as
$(\beta\alpha)(t) = (E\beta)(t)$  %, for all trees with $r(t) \leq q$
and applying the product operation \cite{Butcher2008_book}.
For trees up to order five these are tabulated in Table~\ref{tab:effective_OCs_on_alpha} (and also in \cite[Sec~389]{Butcher2008_book}).
In general, the effective order conditions allow more degrees of
freedom on methods than the classical order conditions.
Note that these conditions match the classical order conditions up to
second order.
\begin{remark}\label{rem:talltrees}
	The effective order conditions of the main method for the ``tall" trees 
	$t_1, t_2, t_4, t_8, t_{17}, \dots$ match the classical order conditions
	and these are precisely the order conditions for linear problems.
	This follows from inductive application of the product operation \cite{Butcher2008_book}
	on the tall trees.  Therefore, methods of effective order $q$ have order at least
        $q$ for linear problems.
\end{remark}

\begin{table}
  \setlength{\extrarowheight}{3pt}
  \centering
  \begin{tabular}{lll}
    \hline
    $q$\;  &  Effective order conditions \\[2pt]
    \hline
    $1$  &
            $\alpha_1  = 1$. \\[2pt]
    \hdashline[2pt/3pt]
    $2$  &
            $\alpha_2  = \tfrac{1}{2}$.  \\[2pt]
    \hdashline[2pt/3pt]
    $3$  &
            $\alpha_3  = \tfrac{1}{3} + 2\beta_2$,  \quad
            $\alpha_4  = \tfrac{1}{6}$.  \\[2pt]
    \hdashline[2pt/3pt]
    $4$  &  \multicolumn{2}{l}{%
            $\alpha_5  = \tfrac{1}{4} + 3\beta_2 + 3\beta_3$, \quad
            $\alpha_6  = \tfrac{1}{8} + \beta_2 + \beta_3 + \beta_4$, \quad
            $\alpha_7  = \tfrac{1}{12} +\beta_2 - \beta_3 + 2\beta_4$, \quad
            $\alpha_8  = \tfrac{1}{24}$.}  \\[2pt]
    \hdashline[2pt/3pt]
    $5$  &
            $\alpha_9  = \tfrac{1}{5} + 4\beta_2 + 6\beta_3 + 4\beta_5$,
         &  $\alpha_{10} = \tfrac{1}{10} + \tfrac{5}{3}\beta_2 - 2\beta_2^{2} + \tfrac{5}{2}\beta_3 + \beta_4 + \beta_5 + 2\beta_6$, \\
         &  \hspace*{-4pt}$\alpha_{11} = \tfrac{1}{15} + \tfrac{4}{3}\beta_2 + \tfrac{1}{2}\beta_3 + 2\beta_4 + 2\beta_6 + \beta_7$,\!\!\!
         &  $\alpha_{12} = \tfrac{1}{30} + \tfrac{1}{3}\beta_2 - 2\beta_2^{2} + \tfrac{1}{2}\beta_3 + \tfrac{1}{2}\beta_4 + \beta_6 + \beta_8$, \\
         &  \hspace*{-4pt}$\alpha_{13} = \tfrac{1}{20} + \tfrac{2}{3}\beta_2 - \beta_2^{2} + \beta_3 + \beta_4 + 2\beta_6$,
         &  $\alpha_{14} = \tfrac{1}{20} + \beta_2 + 3\beta_4 - \beta_5 + 3\beta_7$, \\
         &  \hspace*{-4pt}$\alpha_{15} = \tfrac{1}{40} + \tfrac{1}{3}\beta_2 + \tfrac{3}{2}\beta_4 - \beta_6 + \beta_7 + \beta_8$,
         &  $\alpha_{16} = \tfrac{1}{60} + \tfrac{1}{3}\beta_2 - \tfrac{1}{2}\beta_3 + \beta_4 - \beta_7 + 2\beta_8$, \quad
            $\alpha_{17} = \tfrac{1}{120}$.\!\!\! \\
    \end{tabular}
    \caption{Effective order five conditions on $\alpha$ (main
      method $M$) in terms of order conditions on $\beta$
      (starting method $S$).
      See also \cite[Sec~389]{Butcher2008_book}.
      Recall that $\alpha_i$ and $\beta_i$
      are the elementary weights associated with the index $i$ in
      Table~\ref{tab:elementary_weights}.
      We assume that $\beta_1=0$ (see
      Section~\ref{subsubsec:Main_starting_conditions}).}
    \label{tab:effective_OCs_on_alpha}
\end{table}
%Finally, we use the abbreviation RK($s$,$q$,$p$) for an $s$-stage Runge-Kutta method of effective order $q$ and classical order $p$.

\subsubsection{Order conditions of the main and starting methods}\label{subsubsec:Main_starting_conditions}
As recommended in \cite{Butcher2008_book},
we consider the $\beta_{i}$ as free
parameters when determining the $\alpha_i$.
The relationship in Table~\ref{tab:effective_OCs_on_alpha} between the
$\alpha_i$ and $\beta_i$ is mostly linear (although there are a few
$\beta_2^2$ terms).
It is thus straightforward to (mostly) isolate the equations for $\alpha_i$
and determine the $\beta_i$ as linear combination of the $\alpha_i$.
This separation provides maximal degrees of freedom and minimizes the number of
constraints when constructing the method $M$.
The resulting effective order conditions for the main method $M$ are given
in Table~\ref{tab:effective_OCs} (up to effective order five).
For a specified classical and effective order, these are the equality constraints
$\Phi(K)$ in the optimization problem \eqref{eq:SSP_opt} for method $M$.

Constructing the main method $M$ then determines the $\alpha$ values
and we obtain a set of order conditions on $\beta$ (for that
particular choice of $M$).  These are given in the right-half of
Table~\ref{tab:effective_OCs}.
%The order conditions for the starting method $S$ are also given in the table.
We can also find the order conditions of $S^{-1}$ in terms of the
$\beta_i$ (see \cite[Table~386(III)]{Butcher2008_book}).
We note that increasing the classical order of the main method results
in setting more of the $\beta_i$ to zero.
%The classical order of the main method $M$ is increased by setting $\beta_i$
%to zero.
%Essentially, for a given effective order $q$ if all $\beta_i$ are zero, then
%the main method has classical order $q$.

Tables~\ref{tab:effective_OCs_on_alpha} and~\ref{tab:effective_OCs}
both assume that $\beta_1=0$ (i.e., the starting and stopping methods
perturb the solution but do not advance the solution in time).  This
assumption is without loss of generality following \cite[Lemma
389A]{Butcher2008_book}, the proof of which shows that we can always find 
starting procedures with $\beta_1 = 0$ for which the main method has 
effective order $q$, whenever this holds for a starting method with 
$\beta_1 \neq 0$.
%the proof of which shows that if a method $M$
%has effective order $p$ with particular starting and stopping methods
%(for which $\beta_1 \neq 0$), then $M$ is also effective order $p$
%with another equivalent pair of starting and stopping methods which do have
%$\beta_1=0$.

\begin{table}
  \centering
    \begin{tabular}{M{2mm}M{2mm}M{68mm}M{67mm}}
    		\hline
        $q$ & $p$ & Order conditions for main method $M$ & Order conditions for starting method $S$ \nline
        \hline
        \multirow{1}{*}{3} & \multirow{1}{*}{2} & {\small $\alpha_1 = 1$, $\alpha_2 = \frac{1}{2}$, $\alpha_4 = \frac{1}{6}$.} & {\small $\beta_1 = 0$, $\beta_2 = - \frac{1}{6} + \frac{1}{2}\alpha_3$.}\nline
    \hdashline[2pt/3pt]
    %\hline
        \multirow{3}{*}{4} & \multirow{3}{*}{2} & {\small $\alpha_1 = 1$, $\alpha_2 = \frac{1}{2}$, $\alpha_4 = \frac{1}{6}$,} & {\small $\beta_1 = 0$, $\beta_2 = - \frac{1}{6} + \frac{1}{2}\alpha_3$,}\nline
        & & {\small $\frac{1}{4} - \alpha_3 + \alpha_5 - 2\alpha_6 + \alpha_7 = 0$, $\alpha_8 = \frac{1}{24}$.} & {\small $\beta_3 = \frac{1}{12} - \frac{1}{2}\alpha_3 + \frac{1}{3}\alpha_5$, $\beta_4 = - \frac{1}{24} - \frac{1}{3}\alpha_5 + \alpha_6$.} \nline
    \hdashline[2pt/3pt]
    %\hline
        \multirow{3}{*}{4} & \multirow{3}{*}{3} & {\small $\alpha_1 = 1$, $\alpha_2 = \frac{1}{2}$, $\alpha_3 = \frac{1}{3}$, $\alpha_4 = \frac{1}{6}$,} & {\small $\beta_1 = 0$, $\beta_2 = 0$, $\beta_3 = - \frac{1}{12}  + \frac{1}{3}\alpha_5$,} \nline
        & & {\small $\frac{1}{12} - \alpha_5 + 2\alpha_6 - \alpha_7 = 0$, $\alpha_8 = \frac{1}{24}$.} & {\small $\beta_4 = - \frac{1}{24} - \frac{1}{3}\alpha_5 + \alpha_6$.} \nline
    \hdashline[2pt/3pt]
    %\hline
        \multirow{8}{*}{5} & \multirow{8}{*}{2} & {\small $\alpha_1 = 1$, $\alpha_2 = \frac{1}{2}$, $\alpha_4 = \frac{1}{6}$, $\alpha_8 = \frac{1}{24}$, $\alpha_{17} = \frac{1}{120}$,} & {\small $\beta_1 = 0$, $\beta_2 = - \frac{1}{6} + \frac{1}{2}\alpha_3$,} \nline
        & & {\small $\frac{1}{4} - \alpha_3 + \alpha_5 - 2\alpha_6 + \alpha_7 = 0$,} & {\small $\beta_3 = \frac{1}{12} - \frac{1}{2}\alpha_3 + \frac{1}{3}\alpha_5$, $\beta_4 = -\frac{1}{24} - \frac{1}{3}\alpha_5 + \alpha_6$} \nline
        & & {\small $\frac{1}{4}\alpha_9-\alpha_{10}+\alpha_{13}=\beta_2^{2}$, \: $\beta_2 = - \frac{1}{6} + \frac{1}{2}\alpha_3$,} & {\small $\beta_5 = -\frac{1}{120} + \frac{1}{4}\alpha_3 - \frac{1}{2}\alpha_5 + \frac{1}{4}\alpha_9$,} \nline
        & & {\small $\frac{3}{10} - \frac{3}{2}\alpha_3 + \alpha_5 + \frac{1}{2}\alpha_9 - 3\alpha_{10} + 3\alpha_{11} - \alpha_{14} = 6\beta_2^{2}$,} & {\small $\beta_6 = \frac{7}{720} + \beta_2^{2} + \frac{1}{12}\alpha_3 - \frac{1}{2}\alpha_6 - \frac{1}{8}\alpha_9 + \frac{1}{2}\alpha_{10}$,} \nline
        & & {\small $\frac{1}{15} - \frac{1}{2}\alpha_3 + \alpha_6 + \frac{1}{2}\alpha_9 - 2\alpha_{10} + \alpha_{11} + \alpha_{12} - \alpha_{15} = 2\beta_2^{2}$,} & {\small $\beta_7 = \frac{8}{45} - 2\beta_2^{2} - \frac{7}{12}\alpha_3 + \frac{1}{2}\alpha_5 - \alpha_6 + \frac{1}{4}\alpha_9 - \alpha_{10} + \alpha_{11}$,} \nline
        & & {\small $\frac{19}{60} - \alpha_3 + \alpha_5 - 2\alpha_6 + \alpha_{11} - 2\alpha_{12} + \alpha_{16} = 4\beta_2^{2}$.} & {\small $\beta_8 = -\frac{1}{120} + \beta_2^{2} + \frac{1}{8}\alpha_9 - \frac{1}{2}\alpha_{10} + \alpha_{12}$.} \nline
    \hdashline[2pt/3pt]
    %\hline
        \multirow{7}{*}{5} & \multirow{7}{*}{3} & {\small $\alpha_1 = 1$, $\alpha_2 = \frac{1}{2}$, $\alpha_3 = \frac{1}{3}$, $\alpha_4 = \frac{1}{6}$, $\alpha_8 = \frac{1}{24}$,} & {\small $\beta_1 = 0$, $\beta_2 = 0$, $\beta_3 = -\frac{1}{12} + \frac{1}{3}\alpha_5$} \nline
        & & {\small $\alpha_{17} = \frac{1}{120}$, $\frac{1}{12} - \alpha_5 + 2\alpha_6 - \alpha_7 = 0$,} & {\small $\beta_4 = -\frac{1}{24} - \frac{1}{3}\alpha_5 + \alpha_6$,} \nline
        & & {\small $\frac{1}{4}\alpha_9 - \alpha_{10} + \alpha_{13} = 0$,} & {\small $\beta_5 = \frac{3}{40} - \frac{1}{2}\alpha_5 + \frac{1}{4}\alpha_9$,} \nline
        & & {\small $\frac{1}{5} - \alpha_5 - \frac{1}{2}\alpha_9 + 3\alpha_{10} - 3\alpha_{11} + \alpha_{14} = 0$,} & {\small $\beta_6 = \frac{3}{80} - \frac{1}{2}\alpha_6 - \frac{1}{8}\alpha_9 + \frac{1}{2}\alpha_{10}$,} \nline
        & & {\small $\frac{1}{10} - \alpha_6 - \frac{1}{2}\alpha_9 + 2\alpha_{10} - \alpha_{11} - \alpha_{12} + \alpha_{15} = 0$,} & {\small $\beta_7 = -\frac{1}{60} + \frac{1}{2}\alpha_5 - \alpha_6 + \frac{1}{4}\alpha_9 - \alpha_{10} + \alpha_{11}$,} \nline
        & & {\small $\frac{1}{60} - \alpha_5 + 2\alpha_6 - \alpha_{11} + 2\alpha_{12} - \alpha_{16} = 0$.} & {\small $\beta_8 = -\frac{1}{120} + \frac{1}{8}\alpha_9 - \frac{1}{2}\alpha_{10} + \alpha_{12}$.} \nline
    \hdashline[2pt/3pt]
    %\hline
        \multirow{7}{*}{5} & \multirow{7}{*}{4} & {\small $\alpha_1 = 1$, $\alpha_2 = \frac{1}{2}$, $\alpha_3 = \frac{1}{3}$, $\alpha_4 = \frac{1}{6}$, $\alpha_5 = \frac{1}{4}$,} & {\small $\beta_1 = 0$, $\beta_2 = 0$,} \nline
        & & {\small $\alpha_6 = \frac{1}{8}$, $\alpha_7 = \frac{1}{12}$, $\alpha_8 = \frac{1}{24}$, $\alpha_{17} = \frac{1}{120}$,} & {\small $\beta_3 = 0$, $\beta_4 = 0$,} \nline
        & & {\small $\frac{1}{4}\alpha_9 - \alpha_{10} + \alpha_{13} = 0$,} & {\small $\beta_5 = -\frac{1}{20} + \frac{1}{4}\alpha_9$,} \nline
        & & {\small $\frac{1}{20} + \frac{1}{2}\alpha_9 - 3\alpha_{10} + 3\alpha_{11} - \alpha_{14} = 0$,} & {\small $\beta_6 = -\frac{1}{40} - \frac{1}{8}\alpha_9 + \frac{1}{2}\alpha_{10}$,} \nline
        & & {\small $\frac{1}{40} + \frac{1}{2}\alpha_9 - 2\alpha_{10} + \alpha_{11} + \alpha_{12} - \alpha_{15} = 0$,} & {\small $\beta_7 = -\frac{1}{60} + \frac{1}{4}\alpha_9 - \alpha_{10} + \alpha_{11}$,} \nline
        & & {\small $\frac{1}{60} - \alpha_{11} + 2\alpha_{12} - \alpha_{16} = 0$.} & {\small $\beta_8 = -\frac{1}{120} + \frac{1}{8}\alpha_9 - \frac{1}{2}\alpha_{10} + \alpha_{12}$.} \nline
    \end{tabular}
    \caption{Effective order $q$, classical order $p$ conditions on $ \alpha $ and $ \beta $ for the main and starting methods, $M$ and $S$ respectively.}
    \label{tab:effective_OCs}
\end{table}


