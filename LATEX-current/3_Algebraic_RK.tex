\section{Algebraic representation of Runge--Kutta methods}\label{sec:Algebraic_RK}

The effective order of a Runge--Kutta method is defined in an abstract algebraic
context introduced by Butcher \cite{Butcher1969} and developed further in
\cite{Butcher1972, Hairer1974, Butcher1996, Butcher1998} and others.
In this section we follow \cite{Butcher2008_book}, reviewing the fundamental
concepts of this representation which are then used to define
effective order methods and their order conditions.


\subsection{Formulation of the Runge--Kutta group}\label{sec:RK_group}

%In this characterization, we remove our attention from the numerical methods themselves and we concentrate on their counterparts in a certain group containing .

For our purposes, an algebraic structure for Runge--Kutta methods is useful if it can describe
the successive application of two methods.
Butcher \cite{Butcher1972} proposed the group $G$ of all real-valued maps on the
set of rooted trees.
Each function $\alpha \in G$ corresponds
to an \emph{equivalence class} of Runge--Kutta methods and maps trees to specific
algebraic expressions in the coefficients
of a Runge--Kutta method, known as \emph{elementary weights}.
%The elementary weight of a tree $t$ is an algebraic expression in the coefficients
%of a Runge--Kutta method.
Two Runge--Kutta methods are equivalent if they have the same values of these expressions.


For every function $\alpha \in G$ we write the values of the
elementary weights as $\alpha_{i} = \alpha(t_{i})$ for trees $t_{i}$
indexed by integer $i$.
The infinite-length vector consisting of $\alpha_i$, $i = 0, 1, 2,\ldots$ is the B-series
of the corresponding Runge--Kutta method and is related to Taylor expansions of the numerical solution of the Runge--Kutta method \cite{Hairer1974, Butcher2008_book}.
Table~\ref{tab:elementaryWeights} shows these expressions up to fifth order; a recursive formulation can be found in \cite[Definition 312]{Butcher2008_book}.


\begin{table}
  \centering
  \begin{tabular}{ccccccc}
    \cline{1-3}\cline{5-7}
    $i$ & tree $t_i$ & elementary weight & & $i$ & tree $t_i$ & elementary weight \\
    \cline{1-3}\cline{5-7} \\[-10pt]
    0 & $\emptyset$ \hspace{15pt}  & 1 & & 9 & \hspace{15pt} \tree{9} & $\bm{b}^T\bm{c}^4$\\
    1 & \hspace{15pt}  \tree{1} &$\bm{b}^T\bm{e}$ & & 10 & \tree{10} \hspace{15pt} & $\bm{b}^TC^2A\bm{c}$ \\
    2 & \tree{2} \hspace{15pt}  &$\bm{b}^T\bm{c}$ & & 11 & \hspace{15pt} \tree{11} & $\bm{b}^TCA\bm{c}^2$ \\
    3 & \hspace{15pt}  \tree{3} & $\bm{b}^T\bm{c}^2$ & & 12 & \tree{12} \hspace{15pt} & $\bm{b}^TCA^2\bm{c}$ \\
    4 & \tree{4} \hspace{15pt}  & $\bm{b}^TA\bm{c}$ & & 13 & \hspace{15pt} \tree{13} & $\bm{b}^T(A\bm{c})^2$ \\
    5 & \hspace{15pt}  \tree{5} & $\bm{b}^T\bm{c}^3$ & & 14 & \tree{14} \hspace{15pt} & $\bm{b}^TA\bm{c}^3$ \\
    6 & \tree{6} \hspace{15pt}  & $\bm{b}^TCA\bm{c}$ & & 15 & \hspace{15pt} \tree{15} & $\bm{b}^TACA\bm{c}$ \\
    7 & \hspace{15pt}  \tree{7} & $\bm{b}^TA\bm{c}^2$ & & 16 & \tree{16} \hspace{15pt} & $\bm{b}^TA^2\bm{c}^2$ \\
    8 & \tree{8} \hspace{15pt}  & $\bm{b}^TA^2\bm{c}$ & &  17 & \hspace{15pt} \tree{17} & $\bm{b}^TA^3\bm{c}$ \\
  \end{tabular}
  \caption{Elementary weights of trees up to fifth order.
    % By convention $\alpha(\emptyset) = 1$, where
    % $t_{0} = \emptyset$ denotes the empty tree.
    % Colin removed the previous b/c there is no alpha in this table.
    Here matrix $C =\text{diag}(\bm{c})$ and exponents of
    vectors represent component exponentiation.}
  \label{tab:elementaryWeights}
\end{table}

% \subsubsection*{minimalist}
% Given $\alpha, \beta \in G$, the product can be defined on each
% tree by performing a certain partitioning of the tree and computing
% over the resulting forest \cite{Butcher2008_book}.  The product of
% $\alpha\beta$ corresponds to the application of one Runge--Kutta
% method followed by another.  This multiplicative binary operation
% makes $G$ a group.


Suppose $\alpha, \beta \in G$ correspond to Runge--Kutta methods $A$ and $B$ respectively.
A multiplicative group operation $\alpha\beta$ can be defined
by partitioning the input tree and
%on each tree by performing a certain partitioning of the tree and
computing over the resulting forest \cite{Butcher2008_book}.
This product $\alpha\beta$ corresponds to the application
of method $A$ followed by method $B$ and resulting to a method
$BA$.\footnote{We write $BA$ to mean method $A$ followed by method $B$
  (following matrix and operator ordering convention) but we use the
  ordering $\alpha\beta$ to match the convention in
  \cite{Butcher2008_book}.}
\colintodo{clean up paper: want $\alpha\beta$ and $MS$.}
%Runge--Kutta method followed by another.
The product is defined by
\begin{equation}\label{eq:Group_operation}
  (\alpha\beta)(t) = \sum_{w \lhd t} \left(\prod_{v \in t \setminus w} \alpha(v)\beta(w)\right),
\end{equation}
where $w \lhd t$ indicates a subtree of $t$ which includes the
root of $t$ and $w \setminus t$ indicates the forest induced
by removing $w$ from $t$ \cite{Butcher2008_book}.
Multiplicity in choosing $w$ must also be accounted for.
The following example demonstrates the idea of this product:

%\subsubsection*{longer version}
% The group multiplicative operation $\alpha\beta$ corresponds to the
% application of one Runge--Kutta method followed by another.
% It is defined as follows.
% \begin{definition}\label{def:Group_operation}
% 	Let $\alpha$ and $\beta$ be two members of $G$, mapping trees to real numbers. Then for every tree $t$, we define their product by
% 	\begin{equation}\label{eq:Group_operation}
% 		(\beta\alpha)(t) = \sum_{w \lhd t} \prod_{v \in t \setminus w} \alpha(v)\beta(w),
% 	\end{equation}
%         where $w \lhd t$ indicates a subtree of $t$ which includes the
%         root of $t$ and $w \setminus t$ indicates the forest induced
%         by removing $w$ from $t$.
%         Multiplicity in choosing $w$ must also be accounted for.
% \end{definition}
% See \cite{Butcher2008_book} for details.  This product is perhaps best
% understood via an example.

\begin{example}\label{ex:tree_partition}
	Table~\ref{tab:tree_partition} shows the partition of the five-vertex tree $t_{11}$ to all possible rooted subtrees. Based on this partition, we apply \eqref{eq:Group_operation} to find that the product of two functions in $G$ on tree $t_{11}$ is given by
$(\alpha\beta)(t_{11}) = \alpha_{11} + \alpha_1\alpha_3\beta_1 + (\alpha_1^{3} + \alpha_3)\beta_2 + \alpha_1^{2}\beta_3 + 2\alpha_1^{2}\beta_4 + 2\alpha_1\beta_6 + \alpha_1\beta_7 + \beta_{11}.$

\begin{table}
	\centering
    \begin{tabular}{c cc|c|c|c|c|c|c|c|c|c}
        \multirow{2}{*}{\begin{biggertrees}\treecell{$\tree{11}$}{$t_{11}$}\end{biggertrees}} & & $w$ & $\emptyset$ & $\tree{1}$ & $\tree{2}$ & $\tree{2}$ & $\tree{3}$ & $\tree{4}$ & $\tree{6}$ & $\tree{7}$ & $\tree{11}$ \\[3pt]
        \cline{3-12}
        & & $t_{11} \setminus w$ & \rowscell{$\tree{11}$}{} & \rowscell{$\tree{1}$}{$\tree{3}$} & \rowscell{$\tree{3}$}{ } & \rowscell{$\tree{1} \quad \tree{1}$}{$\tree{1}$} & \rowscell{$\tree{1} \quad \tree{1}$}{ } & \rowscell{$\tree{1} \quad \tree{1}$}{$(\times2)$} & \rowscell{$\tree{1}$}{$(\times2)$} & \rowscell{$\tree{1}$}{ } & \rowscell{$\emptyset$}{ }
    \end{tabular}
    \vspace{5pt}
    \caption{Partitions of tree $t_{11}$ to all
      possible subtrees $w$ and the
      corresponding forests $t_{11} \setminus w$.
      Multiplicity is indicated with $(\times2)$.}
    \label{tab:tree_partition}
	\end{table}
\end{example}


\subsection{Algebraic interpretation of order}\label{sec:Algebraic_order}

If two Runge--Kutta methods map to the same element of $G$ then they
are essentially the same method (up to reducibility).
This is overly restrictive for practical purposes and we want a way to
discuss equivalence of methods up to a particular order of accuracy.

\begin{definition}\label{def:Equivalent_methods}
	Two Runge--Kutta methods
        are equivalent up to order $p$ if their corresponding elements in $G$, $\alpha$ and $\beta$ respectively, satisfy
	\begin{displaymath}
		\alpha(t) = \beta(t), \; \text{for every tree $t$ with $r(t) \leq p$},
	\end{displaymath}
	%Assume that $m$ and $\widetilde{m}$ are Runge-Kutta methods where the corresponding members of $G_1$ are $\alpha$ and $\widetilde{\alpha}$.
        where $r(t)$ denotes the order of the tree (number of vertices).
\end{definition}

In this sense, methods have inverses: the product of $\alpha^{-1}$ and
$\alpha$ must match the identity method up to order $p$
\cite{Butcher2008_book}.
Classical order follows from comparing a method with the special group
element $E \in G$ which advances the exact solution by one step
\cite{Butcher2008_book}.
All of this can be made considerably more precise using quotient
groups of $G$ \cite{Butcher2008_book}.


\begin{example}\label{ex:FE_inv_2}
  Consider the forward Euler method \eqref{eq:forwardEuler}.
  %$\bm{u}^{n+1} = \bm{u}^{n} + \Dt \bm{F}(\bm{u}^{n})$.
  To find an inverse, we seek a method that undoes the work of this method,
  recovering $\bm{u}^{n}$ from $\bm{u}^{n+1}$;
  one approach is to solve for $\bm{u}^{n}$, obtaining the backward Euler method
  with a time-step of $-\Dt$.
  Alternatively, let $\alpha \in G$ correspond to the forward Euler method and by
  \eqref{eq:Group_operation}, we have
  %\cite[Sec~382]{Butcher2008_book}, we have
  $(\alpha\alpha^{-1})(t_1) = \alpha(t_1) + \alpha^{-1}(t_1) = 0$,
  so any $\alpha^{-1}$ with $\alpha^{-1}(t_1) = -1$ will do.
  For example, the forward Euler method with size $-\Delta t$ is also an
  inverse (up to order $1$).
\end{example}
This example demonstrates that inverse methods up to order $p$ are not
unique and inverse methods of explicit methods need not be implicit.


\subsection{Effective order}\label{sec:Effective_order}

Effective order is achieved by using a \emph{starting method} $S$
followed by a \emph{main method} $M$
and then a \emph{finishing method} $S^{-1}$.
%: i.e., the inverse of $S$ which
%annihilates the work of the starting method (up to order $q$).
%The methods $S$ and $S^{-1}$ make the method $M$ effectively of higher order than its classical order.
We denote by $\alpha$ and $\beta$ the functions in group $G$ associated with the methods $M$ and $S$, respectively.
%Thus $\beta^{-1}$ corresponds to Runge--Kutta method $S^{-1}$ that annihilates the work of the starting method $S$ (up to order $q$).
The successive use of these three methods results in a method $P = S^{-1}MS$, of which the corresponding function in $G$ is $\beta\alpha\beta^{-1}$.
We want $P$ to have order $q$ whereas $M$ might have lower classical
order $p < q$.
In terms of functions in group $G$ this is leads to the following definition of the effective order of the Runge--Kutta method $M$.
\begin{definition}\cite{Butcher1987_book}\label{def:Effective_order}
  % Let $\alpha$ and $\beta be the corresponding functions in $G$ of a
  % Runge--Kutta methods $M$ and $S$, respectively.
  Suppose $M$ is a Runge--Kutta method with corresponding $\alpha \in G$.
  Then the method $M$ is of effective order $q$ if there exists a method
  $S$ (with corresponding $\beta \in G$) such that
	\begin{equation}\label{eq:Effective_order_1}
		(\beta\alpha\beta^{-1})(t) = E(t), \; \text{for every tree with $r(t) \leq q$,}
	\end{equation}
        where $\beta^{-1}$ is an inverse of $\beta$ up to order $q$.
\end{definition}
The practical benefit of methods of effective order results from the
observation that only $M$ need be used repeatedly.
\yiannistodo{move this result ot section 5}
\colintodo{leave here for now please}
\begin{result}
  Let method $M$ have effective order $q$ with associated methods $S$,
  $S^{-1}$.  Let $P=S^{-1}MS$ then
  %With methods $M$, $S$, $S^{-1}$, and $P$ as above we have
  $$P^n = (S^{-1}MS)^n = (S^{-1}MS) \cdots (S^{-1}MS) (S^{-1}MS)
        = S^{-1} M^n S.$$
The starting method is applied at the beginning without advancing the
solution.
Instead, it introduces a perturbation on the solution.
The main method \( M \) is then used \( n \) times and finally the
finishing method is used to correct the solution.
\end{result}



\subsubsection{Effective order conditions}\label{sec:effOrderCond}

For the main method $M$ to be effective order $q$ its coefficients must satisfy a series of algebraic conditions coming from each tree in Definition~\ref{def:Effective_order}.
That is, the Runge--Kutta method $M$ corresponding to the function $\alpha$ must satisfy
\emph{effective order conditions} relative to the order conditions of the
method $S$ corresponding to the function $\beta$.
We rewrite \eqref{eq:Effective_order_1} as
$(\beta\alpha)(t) = (E\beta)(t), \; \text{for all trees with $r(t) \leq q$,}$
and using the product operation \eqref{eq:Group_operation}, we can find expressions for each tree $t$ with $r(t) \leq q$.
For trees up to order five these are tabulated in \cite[Table~3.89]{Butcher2008_book}
and in Table~\ref{tab:effectiveOCs}.
Note that these conditions match the classical order conditions up to
second order.
Note also that for the tall trees $t_1, t_4, t_8, t_{17}, \dots$ the
effective order conditions of the main method match the classical
order conditions \cite{Butcher2008_book}.
\begin{table}
	%\vspace*{-5ex}
	%\setlength{\columnseprule}{0.4pt}
	%\setlength{\columnsep}{10pt}
	\begin{multicols}{2}
		\begin{align*}
			% tfrac used here to make smaller fractions
    		\alpha_1  &= 1 \\
     		\alpha_2  &= \tfrac{1}{2} \\
    		\alpha_3  &= \tfrac{1}{3} + 2\beta_2 \\
    		\alpha_4  &= \tfrac{1}{6} \\
    		\alpha_5  &= \tfrac{1}{4} + 3\beta_2 + 3\beta_3 \\
    		\alpha_6  &= \tfrac{1}{8} + \beta_2 + \beta_3 + \beta_4 \\
    		\alpha_7  &= \tfrac{1}{12} +\beta_2 - \beta_3 + 2\beta_4 \\
    		\alpha_8  &= \tfrac{1}{24}
    	\end{align*}
    	\vfill
    	\columnbreak
    	\begin{align*}
    		\alpha_9  &= \tfrac{1}{5} + 4\beta_2 + 6\beta_3 + 4\beta_5 \\
    		\alpha_{10} &= \tfrac{1}{10} + \tfrac{5}{3}\beta_2 - 2\beta_2^{2} + \tfrac{5}{2}\beta_3 + \beta_4 + \beta_5 + 2\beta_6 \\
    		\alpha_{11} &= \tfrac{1}{15} + \tfrac{4}{3}\beta_2 + \tfrac{1}{2}\beta_3 + 2\beta_4 + 2\beta_6 + \beta_7 \\
    		\alpha_{12} &= \tfrac{1}{30} + \tfrac{1}{3}\beta_2 - 2\beta_2^{2} + \tfrac{1}{2}\beta_3 + \tfrac{1}{2}\beta_4 + \beta_6 + \beta_8 \\
    		\alpha_{13} &= \tfrac{1}{20} + \tfrac{2}{3}\beta_2 - \beta_2^{2} + \beta_3 + \beta_4 + 2\beta_6 \\
    		\alpha_{14} &= \tfrac{1}{20} + \beta_2 + 3\beta_4 - \beta_5 + 3\beta_7 \\
    		\alpha_{15} &= \tfrac{1}{40} + \tfrac{1}{3}\beta_2 + \tfrac{3}{2}\beta_4 - \beta_6 + \beta_7 + \beta_8 \\
    		\alpha_{16} &= \tfrac{1}{60} + \tfrac{1}{3}\beta_2 - \tfrac{1}{2}\beta_3 + \beta_4 - \beta_7 + 2\beta_8 \\
    		\alpha_{17} &= \tfrac{1}{120}
    	\end{align*}
        \vfill
    \end{multicols}
    \caption{Effective order five conditions of the main and starting methods $M$ and $S$.}
    \label{tab:effectiveOCs}
\end{table}
Finally, we use the abbreviation RK($s$,$q$,$p$) for an $s$-stage Runge-Kutta method of effective order $q$ and classical order $p$.


\subsection{Constructing effective order methods}
The approach we adopt is to consider the $\beta_{i}$ as free
parameters when determining the $\alpha_i$.
The relationship in Table~\ref{tab:effectiveOCs} between the
$\alpha_i$ and $\beta_i$ is mostly linear (although there are a few
$\beta_2^2$ terms).
It is thus straightforward to isolate the equations for $\alpha_i$,
determining the $\beta_i$ as linear combination of the $\alpha_i$, and
separate the effective order conditions into conditions on main method
$M$ and starting method $S$.
This provides maximal degrees of freedom and minimizes the number of
constraints when constructing the method $M$.
Then when all functions on $\alpha$ are found, the relative order
conditions on $\beta$ can be obtained.

The resulting effective order conditions for main method $M$ are given
in Table~\ref{tab:Effective_oc} (up to effective order five).
The order conditions for the starting method $S$ are also given.
We can also find the order conditions of $\beta^{-1}$ in terms of the
$\beta_i$ (see \cite[Table~386(III)]{Butcher2008_book}).

Note that we follow the common convention is to set $\beta_1=0$
\cite{Butcher2008_book}, i.e., the starting and finishing methods
perturb the solution but do not advance the solution in time
(Tables~\ref{tab:effectiveOCs} and~\ref{tab:Effective_oc} both make
this assumption).


\begin{table}[htb]
	\centering
    \begin{tabular}{M{2mm}|M{2mm}|M{68mm}|M{67mm}}
        q & p & Order conditions for the main method $M$ & Order conditions for the starting method $S$ \nline
        \hline
        \multirow{1}{*}{3} & \multirow{1}{*}{2} & {\small $\alpha_1 = 1$, $\alpha_2 = \frac{1}{2}$, $\alpha_4 = \frac{1}{6}$.} & {\small $\beta_1 = 0$, $\beta_2 = - \frac{1}{6} + \frac{1}{2}\alpha_3$.}\nline
        \hline
        \multirow{3}{*}{4} & \multirow{3}{*}{2} & {\small $\alpha_1 = 1$, $\alpha_2 = \frac{1}{2}$, $\alpha_4 = \frac{1}{6}$,} & {\small $\beta_1 = 0$, $\beta_2 = - \frac{1}{6} + \frac{1}{2}\alpha_3$,}\nline
        & & {\small $\frac{1}{4} - \alpha_3 + \alpha_5 - 2\alpha_6 + \alpha_7 = 0$, $\alpha_8 = \frac{1}{24}$.} & {\small $\beta_3 = \frac{1}{12} - \frac{1}{2}\alpha_3 + \frac{1}{3}\alpha_5$, $\beta_4 = - \frac{1}{24} - \frac{1}{3}\alpha_5 + \alpha_6$.} \nline
        \hline
        \multirow{3}{*}{4} & \multirow{3}{*}{3} & {\small $\alpha_1 = 1$, $\alpha_2 = \frac{1}{2}$, $\alpha_3 = \frac{1}{3}$, $\alpha_4 = \frac{1}{6}$,} & {\small $\beta_1 = 0$, $\beta_2 = 0$, $\beta_3 = - \frac{1}{12}  + \frac{1}{3}\alpha_5$,} \nline
        & & {\small $\frac{1}{12} - \alpha_5 + 2\alpha_6 - \alpha_7 = 0$, $\alpha_8 = \frac{1}{24}$.} & {\small $\beta_4 = - \frac{1}{24} - \frac{1}{3}\alpha_5 + \alpha_6$.} \nline
        \hline
        \multirow{8}{*}{5} & \multirow{8}{*}{2} & {\small $\alpha_1 = 1$, $\alpha_2 = \frac{1}{2}$, $\alpha_4 = \frac{1}{6}$, $\alpha_8 = \frac{1}{24}$, $\alpha_{17} = \frac{1}{120}$,} & {\small $\beta_1 = 0$, $\beta_2 = - \frac{1}{6} + \frac{1}{2}\alpha_3$,} \nline
        & & {\small $\frac{1}{4} - \alpha_3 + \alpha_5 - 2\alpha_6 + \alpha_7 = 0$,} & {\small $\beta_3 = \frac{1}{12} - \frac{1}{2}\alpha_3 + \frac{1}{3}\alpha_5$, $\beta_4 = -\frac{1}{24} - \frac{1}{3}\alpha_5 + \alpha_6$} \nline
        & & {\small $\frac{1}{4}\alpha_9-\alpha_{10}+\alpha_{13}=\beta_2^{2}$, \: $\beta_2 = - \frac{1}{6} + \frac{1}{2}\alpha_3$,} & {\small $\beta_5 = -\frac{1}{120} + \frac{1}{4}\alpha_3 - \frac{1}{2}\alpha_5 + \frac{1}{4}\alpha_9$,} \nline
        & & {\small $\frac{3}{10} - \frac{3}{2}\alpha_3 + \alpha_5 + \frac{1}{2}\alpha_9 - 3\alpha_{10} + 3\alpha_{11} - \alpha_{14} = 6\beta_2^{2}$,} & {\small $\beta_6 = \frac{7}{720} + \beta_2^{2} + \frac{1}{12}\alpha_3 - \frac{1}{2}\alpha_6 - \frac{1}{8}\alpha_9 + \frac{1}{2}\alpha_{10}$,} \nline
        & & {\small $\frac{1}{15} - \frac{1}{2}\alpha_3 + \alpha_6 + \frac{1}{2}\alpha_9 - 2\alpha_{10} + \alpha_{11} + \alpha_{12} - \alpha_{15} = 2\beta_2^{2}$,} & {\small $\beta_7 = \frac{8}{45} - 2\beta_2^{2} - \frac{7}{12}\alpha_3 + \frac{1}{2}\alpha_5 - \alpha_6 + \frac{1}{4}\alpha_9 - \alpha_{10} + \alpha_{11}$,} \nline
        & & {\small $\frac{19}{60} - \alpha_3 + \alpha_5 - 2\alpha_6 + \alpha_{11} - 2\alpha_{12} + \alpha_{16} = 4\beta_2^{2}$.} & {\small $\beta_8 = -\frac{1}{120} + \beta_2^{2} + \frac{1}{8}\alpha_9 - \frac{1}{2}\alpha_{10} + \alpha_{12}$.} \nline
        \hline
        \multirow{7}{*}{5} & \multirow{7}{*}{3} & {\small $\alpha_1 = 1$, $\alpha_2 = \frac{1}{2}$, $\alpha_3 = \frac{1}{3}$, $\alpha_4 = \frac{1}{6}$, $\alpha_8 = \frac{1}{24}$,} & {\small $\beta_1 = 0$, $\beta_2 = 0$, $\beta_3 = -\frac{1}{12} + \frac{1}{3}\alpha_5$} \nline
        & & {\small $\alpha_{17} = \frac{1}{120}$, $\frac{1}{12} - \alpha_5 + 2\alpha_6 - \alpha_7 = 0$,} & {\small $\beta_4 = -\frac{1}{24} - \frac{1}{3}\alpha_5 + \alpha_6$,} \nline
        & & {\small $\frac{1}{4}\alpha_9 - \alpha_{10} + \alpha_{13} = 0$,} & {\small $\beta_5 = \frac{3}{40} - \frac{1}{2}\alpha_5 + \frac{1}{4}\alpha_9$,} \nline
        & & {\small $\frac{1}{5} - \alpha_5 - \frac{1}{2}\alpha_9 + 3\alpha_{10} - 3\alpha_{11} + \alpha_{14} = 0$,} & {\small $\beta_6 = \frac{3}{80} - \frac{1}{2}\alpha_6 - \frac{1}{8}\alpha_9 + \frac{1}{2}\alpha_{10}$,} \nline
        & & {\small $\frac{1}{10} - \alpha_6 - \frac{1}{2}\alpha_9 + 2\alpha_{10} - \alpha_{11} - \alpha_{12} + \alpha_{15} = 0$,} & {\small $\beta_7 = -\frac{1}{60} + \frac{1}{2}\alpha_5 - \alpha_6 + \frac{1}{4}\alpha_9 - \alpha_{10} + \alpha_{11}$,} \nline
        & & {\small $\frac{1}{60} - \alpha_5 + 2\alpha_6 - \alpha_{11} + 2\alpha_{12} - \alpha_{16} = 0$.} & {\small $\beta_8 = -\frac{1}{120} + \frac{1}{8}\alpha_9 - \frac{1}{2}\alpha_{10} + \alpha_{12}$.} \nline
        \hline
        \multirow{7}{*}{5} & \multirow{7}{*}{4} & {\small $\alpha_1 = 1$, $\alpha_2 = \frac{1}{2}$, $\alpha_3 = \frac{1}{3}$, $\alpha_4 = \frac{1}{6}$, $\alpha_5 = \frac{1}{4}$,} & {\small $\beta_1 = 0$, $\beta_2 = 0$,} \nline
        & & {\small $\alpha_6 = \frac{1}{8}$, $\alpha_7 = \frac{1}{12}$, $\alpha_8 = \frac{1}{24}$, $\alpha_{17} = \frac{1}{120}$,} & {\small $\beta_3 = 0$, $\beta_4 = 0$,} \nline
        & & {\small $\frac{1}{4}\alpha_9 - \alpha_{10} + \alpha_{13} = 0$,} & {\small $\beta_5 = -\frac{1}{20} + \frac{1}{4}\alpha_9$,} \nline
        & & {\small $\frac{1}{20} + \frac{1}{2}\alpha_9 - 3\alpha_{10} + 3\alpha_{11} - \alpha_{14} = 0$,} & {\small $\beta_6 = -\frac{1}{40} - \frac{1}{8}\alpha_9 + \frac{1}{2}\alpha_{10}$,} \nline
        & & {\small $\frac{1}{40} + \frac{1}{2}\alpha_9 - 2\alpha_{10} + \alpha_{11} + \alpha_{12} - \alpha_{15} = 0$,} & {\small $\beta_7 = -\frac{1}{60} + \frac{1}{4}\alpha_9 - \alpha_{10} + \alpha_{11}$,} \nline
        & & {\small $\frac{1}{60} - \alpha_{11} + 2\alpha_{12} - \alpha_{16} = 0$.} & {\small $\beta_8 = -\frac{1}{120} + \frac{1}{8}\alpha_9 - \frac{1}{2}\alpha_{10} + \alpha_{12}$.} \nline
    \end{tabular}
    \caption{Effective order $q$, classical order $p$ conditions on $ \alpha $ and $ \beta $ for the main and starting methods, $M$ and $S$ respectively.}
    \label{tab:Effective_oc}
\end{table}


\begin{example}\label{ex:Effective_RK32}
\colintodo{remove this example?}
\yiannistodo{I think yes and incluse the remarks made here in subsection 3.4}
	Constructing an $s$-stage method of effective order 3 with classical order 2 requires satisfying the three conditions given in Table~\ref{tab:Effective_oc}. This is one less condition than constructing a method with classical order 3, since $\alpha_3$ is here a free parameter. After choosing a method and thus a value for $\alpha_3$, the order conditions for the starting method are now fixed as well.
\end{example}
