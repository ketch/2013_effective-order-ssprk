\documentclass[12pt]{article}
\usepackage{amsmath,amssymb,amsfonts,amsthm,graphicx,subfigure,multicol}
\usepackage[hmargin=1.0in]{geometry}
\usepackage{empheq}
\usepackage{epstopdf}
\usepackage{bm}
\usepackage{hyperref}
\usepackage{float,topcapt}
\usepackage[usenames,dvipsnames]{color}

\newcommand{\remark}[2]{\vspace{25pt} \noindent \textbf{Remark #1:\newline} \textit{#2}\vspace{15pt}}
\renewcommand{\newline}{\vspace{15pt}\\}
\newcommand{\sspcoef}{\mathcal{C}}
\newcommand{\ceff}{\sspcoef_{\textnormal{eff}}}
\newcommand{\clin}{\sspcoef^{\textnormal{lin}}_{s,q}}

\newcommand{\colin}[1]{\textcolor{blue}{\\\textbf{C: \footnotesize #1}\\}}
\newcommand{\david}[1]{\textcolor{red}{\\\textbf{D: \footnotesize #1}\\}}
\newcommand{\yiannis}[1]{\textcolor{OliveGreen}{\\\textbf{Y: \footnotesize #1}\\}}
\newcommand{\jim}[1]{\textcolor{magenta}{\\\textbf{J: \footnotesize #1}\\}}

\begin{document}

\title{Revision of "Effective order strong stability preserving Runge--Kutta methods"}
\author{Yiannis Hadjimichael, Colin B.~Macdonald, David I.~Ketcheson \\ and  Jim.~H.~Verner}

\maketitle

The authors would like to thank the reviewers for their constructive comments and the
editor for the time invested. 
We have revised the paper according to the comments and suggestions. 
Answers to the comments of the reviewers have been included below.
\vspace{30pt}

\section*{Referee \#1}
\remark{1}
{First of all, I think the title should be slightly modified. 
The concept of \textit{effective order} is different from the concept of \textit{order} 
but the terms should be used in a similar way (\textit{effective order of a RK method, 
a RK method with effective order p, \dots}).
\newline
Thus, the title ``effective order SSP RK methods" is nonsense.
In a similar way, in the third paragraph of the introduction section, the sentence 
``\dots SSP properties of explicit RK methods of effective order. Effective order 
methods use \dots", is incorrect.
The same comment is valid for the title of section 3.}

The title has been changed to ``Explicit strong stability preserving Runge--Kutta methods
with effective order of accuracy". 
Also all instances of ``effective order methods" are now corrected.
Particularly, the title of section 3 is now ``Runge--Kutta methods with effective order 
of accuracy" and lines 2 and 4 are changed to ``methods of effective order" and
``methods with effective order of accuracy", respectively.
The introduction is redone and care has been taken so that ``effective order" 
is correctly used.

\yiannis{Other possible titles may be:\\
``Explicit SSPRK of effective order $q$", ``On the effective order of SSPRK methods",
``The effective order of SSPRK methods".}

\remark{2}{
The introduction should be redone. 
In the introduction section I expect to have an introduction to the problem, the 
limitations of the known results, and a brief explanation of the new approach proposed. 
These lines are not followed by the authors, and after reading the introduction, the 
reader does not get the ideas of the problem and how it is going to be solved.}

Following your suggestions we have rewritten the introduction. 
We emphasize the main problems of our study and how these are being addressed.

\remark{3}{
A spatial discretization of (2.1) gives rise to autonomous ODEs, but the ODE in (2.2) is
non autonomous. 
On the other hand, in [4] the study is done for autonomous problems.
Hence, as the paper is based on the theory in [4], I think it is more convenient to deal
with autonomous problems (ODE (2.2), equations for RK methods, (2.3)).
\newline
Observe too that in the paper, $t$ is used to denote two different concepts: the independent
variable in the ODE (2.2) and the trees (see e.g., (3.1)).}

The equations (2.2), (2.3) and the definition of the RK methods in page 2 have been
changed to the autonomous form. 
Also, after line 5 of Section 2, the following phrase is added:
``Particularly, $\bm{F}$ may be time-dependent, but we can always make a transformation 
to an autonomous form."

\remark{4}{
Section 3: The authors try to summarize in few pages the nontrivial theory of effective
order for RK methods. 
Some definitions are given with few rigor. 
On the other hand, the reader gets lost with the sequence of definitions and notations, 
e.g., in the first paragraph we find: group G, rooted trees, elements of G as equivalence 
classes, elementary weights, \dots.
\newline
Example 3.1 and Table 3.2 do not seem to be relevant for the explanation of effective
order. 
Example 3.3 does not seem to be relevant for the paper either.
\newline
On the other hand, it is not easy for the reader to understand how Table 3.3 should be
used. It is not clear either why in Table 3.4, methods with classical order e.g. 3, have
$\beta_0 = \beta_1 = 0$. \\
As the aim of the authors is to construct methods with effective order p, the authors
should begin with the material in section 3.3.1, and include a previous section to explain
as brief as possible the idea of effective order (composition of methods, starting and
stopping procedures, and so on).
Tables 3.1 and 3.3 can be included, but the authors should stress that, $\alpha_i$ in 
Table 3.3 is the elementary weight associated to the index $i$ in Table 3.1.
\newline
The interested reader can read [4] so see all the details.}

Section 3 has been rearranged. 
Based on your suggestions we briefly review J. Butcher theory about the algebraic representation
of RK methods, focusing on the structure of the RK group which yields the concepts of classical 
and effective order.
Then a section on how to construct RK methods of effective order follows.
Tables 3.1 and 3.2 are included and we add the line:
``Recall that $\alpha_i$ and $\beta_i$ are the elementary weights associated 
with the index $i$ in Table 3.1",
as mentioned.
Table 3.2 and examples in the first part of this section are omitted. 
Finally in line 9 page 6, we add:
``For a fixed classical and effective order, these are the equality constraints 
$\Phi(K)$ in the optimization problem (2.5)"
to explain how Table 3.3 is used in practice.
In line 11 page 6, we add the following:
``The classical order of the main method $M$ is increased by setting $\beta_i$
to zero. 
Essentially, for a given effective order $q$ if all $\beta_i$ are zero, then
the main method has classical order $q$."
This clarifies the relation between the classical order and $\beta_i$.

\remark{5}{
The fact that, for $q \leq 5$, methods with effective order $q$ have order $q$ 
for linear problems is important for inequality (5.1). 
This information is given at the end of section 3.1.1, but I think it should be given as a Remark.}

Thanks for the comment. We state this with a remark at the end of section 3.2.1: 

\textbf{Remark 3.3:} The effective order conditions of the main method for the ``tall" trees 
$t_1, t_4, t_8, t_{17}, \dots$ match the classical order conditions
and these are precisely the order conditions for linear problems.
This follows directly from inductive application of (3.1)
on the tall trees.

\remark{6}{
Section 4 is a bit messy, and I propose to change the order of the material in it. 
I propose to state Lemma 4.3, then state Lemma 4.2, and then state Theorem 4.1. 
The authors should explain that the proofs will be done later on. 
The authors should stress the beginning of proof of Lemma 4.2 (in a quick reading, 
it is difficult to find it).}

We rearrange the material in this section as suggested.
We also state that the proof of Lemma 4.2 can be found at the end of the section.

\remark{7}{
It is not clear to me the relevance of Remark 4.4 in this paper.}

We have added the following line before Remark 4.4:

``As a direct consequence of Lemma 4.2 we have the following 
remark regarding circle contractive irreducible methods."

We believe that this is a simple application of Lemma 4.2 and Theorem 4.1
from [6] and requires no further proof. In our opinion this remark may be useful for
relevant studies.

\yiannis{The above may seems that we are arguing with the referee, so we should think 
carefully how to  him/her on the importance of the remark (if we decide to keep it). }

\remark{8}{
Proof of Lemma 4.2, Case 2: Just after defining the set $I$, the authors should stress that
$I \neq \emptyset$ because $v_1 = 0$ and $\beta_2 \neq 0$.
The explanation of why $v_1 = 0$ should be included
here.}

The explanation why $v_1 \neq 0$ is now after set $I$ is defined. The text reads:

``	Let the set $I = \{i : v_i = \beta_2\}$. 
Note that $v_1 = 0$ because the first row of matrix $A$ is identically zero.
Since $\beta_2 \neq 0$ and $ \bm{v} \neq \bm{0}$, the set $I$ is not empty.
Then (4.2a) yields, \dots".
	
\remark{9}{
Proof of Lemma 4.2, Case 3: The authors should explain, that, as $v_1 = 0$, the 
indexes $i,j$ are in $\{2,\dots,s\}$.}

Lemma 4.2, Case 3 has been changed to:

``Since $\bm{b} > \bm{0}$, (4.6) implies that $\bm{w}$ contains both positive 
and negative elements. 
Furthermore, $v_1=0$ for any explicit method, thus $w_1=0$.
Then, we can choose $i, j \in \{2, \dots, s\}$ such that $w_i < 0 < w_j$.
By (4.3) $v_i\ne 0$, $v_j\ne 0$, and $v_i\ne v_j$.
Application of Lemma 4.5 reveals that all $v_k$ for 
$k \in\{1,2,\dots,s\}$ must be equal except for one, which is a contradiction
since $v_1 = 0$."

\remark{10}{
The authors should explain the inequality $\sspcoef_{s,q,p} \leq \clin$ in formula (5.1). 
For $q \leq 5$ inequalities (5.1) are true because for $q \leq 5$ methods with effective 
order $q$ have order $q$ for linear problems. 
For $q \geq 5$, they are also true because by Theorem 4.1 $\sspcoef_{s,q,p}  = 0$.}

We agree with this argument, however Remark 3.3 is much stronger. 
For any RK method the effective order conditions $q$ include the classical order $q$ 
conditions for linear problems, i.e. the conditions on the tall trees. 
As we state in Remark 3.3 this can be easily proved by inductively applying the formula 
(3.1) on tall trees. 
Then one can see that for a given order the effective order conditions on tall trees 
match the order conditions for linear problems.

Thus we add the following line before inequalities (5.1):

``From Remark 3.3 and the fact that the ESSPRK($s,q,p$) methods 
form a super class of the SSPRK($s,q$) methods, we have \dots"

\yiannis{Same comment as remark 7.}

\remark{11}{
Legend in Table 5.1: Please, include the value of $\sspcoef^{\textnormal{lin}}_{6,4}$.}

We include $\sspcoef^{\textnormal{lin}}_{6,4}=0.44$.
However, as mentioned in the previous line in Table 5.1 caption, entries in bold correspond to 
the optimal bound $\clin$ for linear problems. Thus, the ESSPRK($6,4,2$) method achieves the 
optimal SSP coefficient.
\pagebreak

\noindent \textbf{\large Minor remarks:}

\remark{1}{
Page 1, line 10 in section 1: the authors say that SSP ``time discretizations (formerly
TVD discretizatons \dots". 
In some contexts, the term TVD discretizations are also used nowadays. 
I would suggest to change the term ``formely" and use ``also known as TVD discretizations" 
(or something similar).}

We follow your suggestion.

\remark{2}{
Section 2: Please, explain the meaning of $A, b$ and $c$.}

The coefficients $A, b$ and $c$ of an RK method are explained after its definition in page 2:

``Such methods are characterized by the coefficient matrix $A = (a_{ij}) \in 
\mathbb{R}^{s \times s}$, the weight vector $\bm{b} = (b_i) \in \mathbb{R}^s$
and the abscissa $\bm{c} = (c_i) \in \mathbb{R}^s$, where 
$c_i = \sum_{j=1}^{i-1}a_{ij}$.
The accuracy and stability of the method depend on the coefficients of the 
Butcher tableau $(A,\bm{b},\bm{c})$ [4]."

We clarify this in Table 3.1 as well.

\remark{3}{
Equations (2.4): Please, explain that $I$ is the identity matrix. Observe that $I$ is used to
denote the identity matrix and certain set in the proof of Lemma 4.2 (Case 2).}

We define that $I$ is the $s \times s$ identity matrix in (2.4).
We also change the set $I$ in proof of Lemma 4.2 to $J$ to remove any ambiguity.

\remark{4}{
Table 5.2, first row: I think is better to write $\tau(t) = (\alpha\beta^{-1})(t)$.}

We follow your suggestion.

\section*{Referee \#2}

\remark{1}{
On page 3, where the effective SSP coefficient is defined, it would be clearer to say
``To allow a fair comparison of explicit methods with different number of stages, we consider \dots". }

We follow your suggestion.

\remark{2}{
On page 5, in Example 3.3, the use of the phrase ``backward Euler" has been used 
(ambiguously) to refer to the backward-in-time Euler in the past, and the clause 
``time-step of $-\Delta t$" makes it sound like it is suggesting that this is indeed 
the ``backward Euler" the authors are talking about. 
However, clearly the authors in this case mean the ``implicit Euler with a time-step 
of $-\Delta t$", so perhaps it would remove the ambiguity to say it this way.}

We agree with your comment, however we decided to remove the particular example 
as suggested by the other referee.

\end{document}
