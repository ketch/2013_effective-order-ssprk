\section{Conclusions}\label{sec:Conclusion}
We used the theory of strong stability preserving time discretizations
with Butcher's algebraic interpretation of order to construct
effective order SSP Runge--Kutta (ESSPRK) methods. 
These methods, when accompanied by starting and stopping
schemes, attain an order of accuracy higher than their (classical) order.
We proposed a new choice of starting and stopping methods to allow the
overall procedure to be SSP.
We proved that explicit Runge--Kutta methods with strictly positive 
weights have at most effective order four. 
This extends the barrier already known in the case of classical order
explicit SSPRK methods.

ESSPRK methods of effective order three and four
were constructed by numerical optimization.
Most of the methods found were optimal because they achieved
the upper bound on the SSP coefficient known from linear
problems.
Also, despite the non-existence of four-stage, four-order explicit SSPRK methods, 
we found effective order four methods with four stages (of classical 
order two and three). 
We performed numerical tests which confirmed the accuracy and
SSP properties of the ESSPRK methods.
%Interestingly, compared to what is observed with standard SSPRK schemes,
%there was much closer agreement between the (theoretical) stability limit
%given by the SSP coefficient and that observed in practice
%(e.g., when measuring total variation of solutions of Burgers' equation).

The ideas here were applied to explicit Runge--Kutta methods, but they
could also be applied to other classes of methods including implicit
Runge--Kutta methods, general linear methods, and Rosenbrock methods.
