\section{Conclusion}\label{sec:Conclusion}
The theory of strong-stability preserving time discretization has been 
combined with the algebraic interpretation of effective order Runge--Kutta 
methods. 
Based on this concept, we succeeded in constructing the first explicit 
effective order SSP Runge--Kutta (ESSPRK) methods. 
These methods when accompanied with relative staring and finishing 
procedures result to schemes that attain higher order than the order of 
the constructing methods. 
It is proved that explicit Runge--Kutta methods with strictly positive 
weights have at most fourth effective order. 
This generates a barrier for explicit effective order SSP methods and 
hence we studied ESSPRK methods of third and fourth effective order.

Optimal ESSPRK methods were constructed and we studied their SSP 
properties in comparison with the classical SSPRK methods for linear and 
nonlinear problems. 
We were able to find families of third effective order ESSPRK methods. 
Also, despite the non existence of four-stage, four-order SSPRK methods, 
we found fourth effective order methods with four stages and of classical 
order two and three. 

Going beyond the initial formulation of the effective order schemes, a 
different choice of the starting and finishing methods allows the 
formation of an overall SSP scheme. 
The starting method not only perturbs but also advances the solution by 
a single time-step.  
Then the main ESSPRK method carries out the rest of the computation. 
At the final step a correction is applied by using a finishing method which 
basically eradicates the accumulated effect of the starting method. 
As a result the numerical approximation has order of accuracy equal to 
the effective order of the main method and thus higher than its classical 
order. 
The starting and finishing methods have minimal computation cost as 
they contain at most two more stages than those of the main method.

Finally, we performed numerical tests on nonlinear problems. In this way 
we showed that the schemes achieved their desired order. 
Moreover, we emphasized the importance of the SSP coefficient as an 
indicator of the maximum possible time-step for the solution of 
hyperbolic pdes. 
As shown in the case of Burgers' equation allowing slightly bigger time-steps 
than the theoretical bounds results in a highly oscillatory numerical 
approximation.  

\subsection{Future Work}\label{subsec:future_work}
The ideas here were applied to explicit Runge--Kutta methods, but they
can be applied to implicit Runge--Kutta methods and General Linear
Methods \cite{Butcher2008_book}.

Also, other possible directions may be the investigation of low-storage 
effective order SSP methods and advancing the SSP theory of effective 
methods. 
Finally, application of the effective order theory on other classes of 
methods, for example Rosenbrock methods, is also worthy of future 
research.