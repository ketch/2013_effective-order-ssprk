\textbf{Copy-paste job from TSRK paper}

\subsection{Buckley--Leverett}

The Buckley--Leverett equation is a model for two-phase flow through
porous media and consists of the conservation law
%\eqref{eq:HCL} with flux function
\begin{equation*}
U_t + f(U)_x = 0, \quad \text{with} \quad
f(U) = \frac{U^2}{U^2 + a(1-U)^2}.
\end{equation*}
We use $a=\frac{1}{3}$ and initial conditions 
%$u(0,x) = \frac{1}{2}H(x-1/2)$
\begin{equation*}
  u(x,0)= \left\{  \begin{array}{ll}
      1 & \mbox{if $x \le \frac{1}{2}$,} \\
      0 & \mbox{otherwise,} \\ \end{array} \right.
\end{equation*}
on $x \in [0,1)$ with periodic boundary conditions.  Our spatial
discretization uses $100$ points and following
\cite{Hundsdorfer/Verwer:2003, Ketcheson2009_article} we use a conservative scheme with Koren limiter.  We
compute the solution until $t_f = \frac{1}{8}$.  For this problem, the
Euler solution is total variation diminishing (TVD) for $\dt \le \DtFE
= 0.0025$ \cite{Ketcheson2009_article}.  As
discussed above, we must also satisfy the SSP time-step restriction
for the starting method.


%The Buckley--Leverett equation is a model for two-phase flow through
%porous media.  As described in
%\cite{Ferracina/Spijker:2007:SDIRKssp,Ketcheson2009_article}, we use it as a
%test case by computing numerical solutions using the various SSP TSRK
%schemes.  We measure the total variation at each time step and make
%sure it is not increasing (i.e., that the solution has the ``TVD''
%property).  We compute the maximal TVD time step as $\dt =
%\sigma_{\text{BL}} \DtFE$ and compare it to $\dt = \sspcoef \DtFE$.



Figure~\ref{fig:bucklev} shows typical solutions using an TSRK scheme
with timestep $\dt = \sigma \DtFE$.  Table~\ref{tab:bucklev_TVD} shows
the maximal TVD time-step sizes, expressed as $\dt =
\sigma_{\text{BL}} \DtFE$, for the Buckley--Leverett test problem.
The results show that the SSP coefficient is a lower bound for what is
observed in practice, confirming the theoretical importance of the SSP
coefficient.

\begin{figure}
  %\centerline{%
  %\includegraphics[width=0.45\textwidth]{figures/bl_tsrk128_s0p9}%
  %\includegraphics[width=0.45\textwidth]{figures/bl_tsrk128_s5p5}%
  %}
  %\centerline{%
  %\includegraphics[width=0.45\textwidth]{figures/bl_tsrk125_s5p2}%
  %\includegraphics[width=0.45\textwidth]{figures/bl_tsrk125_s9}%
  %}
  \centerline{%
  \includegraphics[width=0.45\textwidth]{figures/bl_tsrk85_s3p5}%
  \includegraphics[width=0.45\textwidth]{figures/bl_tsrk85_s5p6}%
  }
  \caption{Two numerical solutions of the Buckley--Leverett test
    problem.  Left: time-step satisfies the SSP time-step restriction
    (TSRK(8,5) using $\dt = 3.5\DtFE$).  Right: time-step
    does not satisfy the restriction ($\dt = 5.6\DtFE$) 
    and visible oscillations have formed, increasing the total variation
    of the solution.}
  \label{fig:bucklev}
\end{figure}


\begin{table}
  \caption{SSP coefficients versus largest time steps exhibiting
    the TVD property ($\dt = \sigma_{\text{BL}} \DtFE$) on
    the Buckley--Leverett example, for
    some of the SSP TSRK($s$,$p$) schemes.  The effective SSP
    coefficient $\ceff$ should be a lower bound for
    $\sigma_{\text{BL}} / s$ and indeed this is observed.
    SSPRK(10,4) \cite{Ketcheson2008_article} is used as the first step
    in the starting procedure.}
  \label{tab:bucklev_TVD}
  \center
  \begin{tabular}{l|cc|cc}
    \hline
    Method & \multicolumn{2}{c|}{theoretical} & \multicolumn{2}{c}{observed} \\
                  & $\sspcoef$ & $\ceff$&  $\sigma_{\text{BL}}$  & $\sigma_{\text{BL}} / s$  \\ \hline
      TSRK(4,4)   & 1.5917   &  0.398     &   2.16 &  0.540   \\
      %TSRK(5,5)   & 1.6213   &  0.324     &   2.53 &  0.506  \\
      TSRK(8,5)   & 3.5794   &  0.447     &   4.41 &  0.551  \\
      TSRK(12,5)  & 5.2675   &  0.439     &   6.97 &  0.581  \\
      TSRK(12,6)  & 4.3838   &  0.365     &   6.80 &  0.567  \\
      TSRK(12,7)  & 2.7659   &  0.231     &   4.86 &  0.405  \\
      TSRK(12,8)  & 0.94155  &  0.0785    &   4.42 &  0.368  \\
      \hline
  \end{tabular}
\end{table}


